\documentclass[12pt]{report}

% Packages nécessaires
\usepackage[utf8]{inputenc}
\usepackage[T1]{fontenc}
\usepackage[english]{babel}
\usepackage{titlesec}
\usepackage{tocloft}
\usepackage{lipsum} % Package de remplissage de texte (à retirer dans le rapport final)
\usepackage{graphicx}
\usepackage{float} % pour l'option de placement [H]
\usepackage[colorlinks=true, linkcolor=blue, citecolor=green]{hyperref}
\usepackage{minted} % pour la coloration syntaxique
\usepackage{wrapfig} % Pour l'enroulement du texte autour des images
\usepackage{array}
\usepackage{colortbl} %Pour les couleurs des colonnes et lignes des tableaux
% Définition de nouveaux types de colonnes centrées
\newcolumntype{C}[1]{>{\centering\arraybackslash}p{#1}}

% Mathématiques
\usepackage{amsmath, amssymb}
\usepackage{amsthm}
\usepackage{mdframed}  % Package pour les cadres

% Environnements pour les théorèmes, définitions et preuves
\newtheorem{theorem}{Théorème}
\newtheorem{definition}{Définition}
\newtheorem{example}{Exemple}
\newtheorem{property}{Propriété}

\newmdtheoremenv{boxedproperty}{Propriété}[section]  % Crée un environnement encadré pour les propriétés

% Configuration de la page
\usepackage[a4paper, left=2.5cm, right=2.5cm, top=2.5cm, bottom=2.5cm]{geometry}

% Pour avoir le petit dessin de l'INSA en bas à droite de chaque page
\usepackage{background}
\backgroundsetup{
scale=1,
angle=0,
opacity=1,
color=black,
contents={\begin{tikzpicture}[remember picture,overlay]
\node at ([xshift=-0.8in,yshift=0.8in] current page.south east) % Adjust the position of the logo.
{\includegraphics[scale=0.8]{images/pattern.png}}; % logo goes here
\end{tikzpicture}}
}

% Configuration des titres des sections et sous-sections
\titleformat{\chapter}[display]
  {\normalfont\huge\bfseries}{\chaptername\ \thechapter}{20pt}{\Huge}
\titleformat{\section}
  {\normalfont\Large\bfseries}{\thesection}{1em}{}
\titleformat{\subsection}
  {\normalfont\large\bfseries}{\thesubsection}{1em}{}

% Numérotation des chapitres en chiffres romains
\renewcommand{\thechapter}{\Roman{chapter}.}
% Numérotation des sections et sous-sections en chiffres arabes
\renewcommand{\thesection}{\arabic{section}.}
\renewcommand{\thesubsection}{\arabic{section}.\arabic{subsection}}

% Configuration de la table des matières
\renewcommand{\cftchapfont}{\bfseries}
\renewcommand{\cftsecfont}{\normalfont}
\renewcommand{\cftsubsecfont}{\normalfont}
\renewcommand{\cftchappagefont}{\bfseries}
\renewcommand{\cftsecpagefont}{\normalfont}
\renewcommand{\cftsubsecpagefont}{\normalfont}
\setlength{\cftbeforetoctitleskip}{0pt}
\setlength{\cftaftertoctitleskip}{10pt}
\renewcommand{\contentsname}{Table des matières}

\begin{document}
    % Page de titre
\begin{titlepage}
    \centering

    % Minipages pour les logos
    \noindent % Assure qu'il n'y a pas d'indentation au début de la ligne
    \begin{minipage}{0.5\textwidth}
        \includegraphics[width=0.5\linewidth]{images/logo_INSA.png} % Ajustez le chemin et la taille
    \end{minipage}%
    \hfill % Assure que les deux minipages seront poussés à l'extrême gauche et droite
    \begin{minipage}{0.5\textwidth}
        \flushright % Alignement à droite dans la minipage
        \includegraphics[width=0.5\linewidth]{images/logo_universite_toulouse.png} % Ajustez le chemin et la taille
    \end{minipage}


    \vspace*{2cm} % Espace vertical de 2 cm
    {\Huge\bfseries Innovative Smart System \par}
    \vspace{1cm}
    {\huge Portefolio\par}
    \vspace{2cm}
    {\Large \textbf{Clément Gauché} \par}
    \vspace{2cm}
    
    \begin{figure}[H]
    \centering
    \includegraphics[width=0.7\textwidth]{images/Logo_ISS_sans_texte.png}
    \end{figure}
    
    \vfill
    {\Large \textbf{Institut National des Sciences Appliquées de Toulouse} \par}
    \vspace{1cm}
    {\large \today \par}
\end{titlepage}

% Résumé concis, souvent utilisé pour donner un aperçu rapide du contenu d'une recherche ou d'un article académique.
\begin{abstract}
    \lipsum[1] % Lorem Ipsum
\end{abstract}

% Table des matières
\tableofcontents

% Introduction
\chapter{Introduction}
Ce portfolio est organisé de la façon suivante.\\
Chaque section correspond a une matière, et est décomposée en sous parties:
\begin{itemize}
    \item Context
    \item Technical summary
    \item Pratical Work
    \item Competences gained
    \item Analysis and remarks
    \item Reflections
\end{itemize} % Hello

% Début du contenu du rapport
\chapter{Courses description}
\section{Layout of the year's courses}
\begin{figure}[H]
    \centering
    \begin{tabular}{|C{2cm}|C{2cm}|C{4cm}|C{6cm}|}
        \hline
        \rowcolor{black!15!white}\textbf{Date} & \textbf{Duration} & \textbf{Context} & \textbf{Content} \\
        \hline
        from 15/06 to 22/09/2024 & 576 hours & Internship at IASI in Romania & Trainee in the network department, Implementation of monitoring functions in a Software Defined Network (SDN) \\
        \hline
        from 30/09/2024 to 29/01/2025 & 68.5 hours & Innovative Project & The guideline industrial project of our year \\
        \hline
        from 30/09 to 15/10/2024 & 21.5 hours & Cloud and Edge computing & Learn and practice about virtualization techniques such as VMs and containers \\
        \hline
        from 30/09 to 03/12/2024 & 7.5 hours & 5G Technologies & Learn about 5G and more generally topics related to cellular networks \\
        \hline
        from 04/10/2024 to 14/01/2025 & 33.5 hours & Service Architecture & Learn about legacy and modern architectures service oriented for software engineering \\
        \hline
        from 07/10 to 14/12/2024 & 18.25 hours & Wireless Sensor Networks & Learn about wireless sensor networks technologies \\
        \hline
        from 15/10 to 09/12/2024 & 14.75 hours & Middleware for IoT & Discover communication protocols for IoT \\
        \hline
        from 15/10 to 24/01/2025 & 34.5 hours & Security for connected objects & Discussion about the need and how to secure protocols for IoT \\
        \hline
        from 07/10 to 17/10/2024 & 10.5 hours & Energy for Connected Objects & Introduction to the different methods to power Energy for Connected Objects \\
        \hline
        
    \end{tabular}
\end{figure}


\begin{figure}[H]
    \centering
    \begin{tabular}{|C{2cm}|C{2cm}|C{4cm}|C{6cm}|}
        \hline
        \rowcolor{black!15!white}\textbf{Date} & \textbf{Duration} & \textbf{Context} & \textbf{Content} \\
        \hline
        from 04/11 to 08/11/2023 & 13.75 hours & Lab at AIME & Discover the creation of a sensor \\
        \hline
        from 12/11 to 18/12/2024 & 39 hours & Microcontrollers Open-Source Hardware and Sensor Introduction & Introduction to microcontrollers programming and implementation of our sensor in a complex circuit \\
        \hline
        from 04/12/2024 to 07/01/2025 & 7.5 hours & Communication protocols for LP-WPAN & Definition of a TCP/IP protocol stack for an LP-WPAN network \\
        \hline
        from 04/12/2024 to 22/01/2025 & 15.75 hours & Embedded IA for IoT & Introduction to the concepts of AI applied to an IoT context \\
        \hline
        from 08/01 to 24/01/2024 & 10.5 hours & Emerging network (SDN, NGN) & Discovery of emerging network paradigms \\
        \hline
        from 05/10 to 31/01/2024 & 6.75 hours & Portfolio & Writing of the document you are actually reading \\
        \hline
    \end{tabular}
    \caption{Table with information about all my projects}
\end{figure}


\newpage
% Apprenticeship at Vitesco and internship at IASI
% \section{Apprenticeship at Vitesco Technologies}
% \subsection{Context}
% Au cours de ma 2ème année d'alternance chez Vitesco Technologies, mes projets se sont concentrés sur du développement logiciel embarqué, et plus préciseement de la conception de drivers generiques pour des microcontroleurs. J'ai par exemple cette année travaillé sur le développement d'un driver PWM pour un nouveau microcontrolleur de la famille Infineon AURIX, le TC4. J'ai également fais des tests d'intégration d'un projet que sur lequel j'avais travailé l'année passée, le Wake-Up Controller.
% Dans le cadre de la validation de mon diplome j'ai du realiser un stage à Iasi, en Roumanie, dans lequel j'ai travaillé à la première activation du harware security manager (HSM) d'un microcontrolleur de la famille Renesas.

% \subsection{Technical Summary}
% \indent \indent Le développement du pilote PWM pour le microcontroleur TC4 d'Infineon est la seule nouvelle tache que j'ai réalisé dans l'équipe Complexe Device Driver (CDD) cette année. J'ai du concevoir un driver générique qui permettrait de controler les signaux PWM de manière simple et efficace. J'ai également réalisé des tests unitaires et d'intégrations pour valider le bon fonctionnement du "Wake-Up Controller", un projet sur lequel j'avais travaillé l'année précédente. La fonction principale de ce module est de réveiller le microcontrolleur principal d'un etat de veille profonde, ou d'un arret complet. Pour ce faire, j'ai conçu et appliqués des scénarios de test pour valider les déclenchements de réveil à partir de signaux analogiques.

% Enfin, j'ai réalisé un stage de 3 mois à Iasi, en Roumanie, dans lequel j'ai travaillé sur l'activation du Hardware Security Manager (HSM) d'un microcontrolleur de la famille Renesas, le RH850. J'ai du comprendre le fonctionnement de ce module, et en effectué la première activation, en se garantissant la possibilité de le déverouiller. Car en théorie, une fois le HSM activé, il n'est plus possible de reprogrammer le microcontrolleur.

% \subsection{Skills Acquired}
% Au cours de mon apprentissage, j'ai développé différentes compétences techniques et professionnelles :
% \begin{table}[H]
%     \centering
%     \arrayrulecolor{black} % Defines the color of the table borders
%     \renewcommand{\arraystretch}{1.5} % Adjusts the vertical spacing between rows
%     \begin{tabular}{|p{3.5cm}|p{8cm}|c|}
%         \hline
%         \rowcolor[gray]{0.8}
%         \textbf{Skill} & \textbf{Description} & \textbf{Level of Mastery} \\
%         \hline
%         Embedded Systems Development & Proficient in designing and implementing drivers for embedded systems (e.g., PWM driver development). & Advanced \\
%         \hline
%         Microcontroller Programming & Advanced knowledge of microcontroller architectures (e.g., Infineon TC4, Renesas RH850). & Advanced \\
%         \hline
%         Software Validation & Expertise in unit, integration, and black-box testing of embedded modules. & Advanced \\
%         \hline
%         Hardware Security Implementation & Hands-on experience activating and configuring Hardware Security Modules (HSM) for secure communication. & Intermediate \\
%         \hline
%         Low-Level Programming & Proficient in writing and debugging low-level code, including linker configuration and interrupt handling. & Advanced \\
%         \hline
%         Collaboration and Project Management & Developed teamwork and coordination skills through international and cross-functional collaboration. & Advanced \\
%         \hline
%         Software Engineering Principles & Applied modular and scalable development practices, including BuildUnit management. & Advanced \\
%         \hline
%         Cross-Functional Communication & Experience conveying technical details to multidisciplinary teams (e.g., firmware, hardware, and testing groups). & Advanced \\
%         \hline
%         Documentation and Reporting & Created technical documentation and test reports for internal validation and project deliverables. & Advanced \\
%         \hline
%         International Work Experience & Gained cultural and professional exposure by working with multinational teams. & Intermediate \\
%         \hline
%     \end{tabular}
%     \caption{Skills Acquired During the Apprenticeship at Vitesco Technologies}
% \end{table}

% \subsection{Analysis and remarks}

% \indent \indent %Avoir fait le choix de l'alternance en 3ème année  m'a permis de découvrir la réalité du monde industriel, et de l'importance des cours qui sont enseignées à l'INSA. J'ai pu à la fois mettre en pratque des connaissances acquises à l'école, tout en gagnant de nouvelles compétences techniques et professionnelles.
% %J'ai pu travailler sur des projets variés, allant du développement de drivers pour microcontrolleurs, à l'activation d'un module de sécurité matérielle sur un microcontrolleur. J'ai également eu l'opportunité de travailler à l'étranger, en Roumanie, ce qui m'a permis de découvrir une nouvelle culture et de travailler avec des équipes internationales.

% Avoir choisi l'alternance en 3A a été une étape déterminante dans mon parcours. Cette expérience m'a permis de comprendre concrètement la réalité du monde industriel et d'apprécier davantage l'importance des cours dispensés à l'INSA. J'ai pu mettre en pratique les connaissances acquises à l'école tout en développant de nouvelles compétences, à la fois techniques et professionnelles. Les projets auxquels j'ai participé ont couvert un large éventail de domaines, allant du développement de drivers pour microcontrôleurs à l'activation d'un module de sécurité matérielle. De plus, mon séjour en Roumanie pour travailler avec les équipes de Iasi m'a offert une immersion culturelle et professionnelle enrichissante, et m'a permis de collaborer avec des spécialistes internationaux.

% Le projet d'activation du Hardware Security Manager (HSM) m'a particulièrement intéréssé, car il m'a confronté aux défis de la conception des systèmes sécurisés et m'a permis de voir comment sont appliqués les principes de sécurité étudiés en cours. J'ai également appris à mieux communiquer et à gérer la coordination de projets avec des équipes dispersées géographiquement, ce qui a renforcé mes compétences en gestion de projet et en collaboration internationale. La prise en main d'un nouveau microcontrôleur, comme le RH850 de Renesas, a représenté une courbe d'apprentissage exigeante, qui a nécessité persévérance et travail d'équipe pour être surmontée.

% Enfin, concilier les exigences académiques et professionnelles a renforcé ma capacité à gérer mon temps et à définir mes priorités. Vous comprenez donc pourquoi je considère mon alternance chez Vitesco Technologies comme un tournant majeur dans mon parcours d'ingénieur : elle m'a permis d'approfondir mon savoir-faire technique, d'élargir ma vision du travail en équipe à l'international et de confirmer mon intérêt pour la cyber et la sécurité.

\section{Apprenticeship at Vitesco Technologies}
\subsection{Context}
\indent \indent During my second year of apprenticeship at Vitesco Technologies, my projects focused on embedded software development, specifically on the design of generic drivers for microcontrollers. For instance, this year I worked on developing a PWM driver for a new microcontroller in Infineon's AURIX family, the TC4. I also carried out integration tests for a project I had worked on the previous year: the Wake-Up Controller.  
As part of my degree requirements, I had to complete an internship in Iasi, Romania, where I worked on the initial activation of the Hardware Security Manager (HSM) for a Renesas microcontroller.

\subsection{Technical Summary}
\indent \indent Developing the PWM driver for Infineon's TC4 microcontroller was the only new task I took on within the Complex Device Driver (CDD) team this year. I had to design a generic driver that would enable simple and efficient control of PWM signals. I also performed unit and integration tests to validate the proper functioning of the \textit{Wake-Up Controller}, a project I had contributed to the previous year. The main function of this module is to wake the main microcontroller from a deep sleep state or from a complete shutdown. To achieve this, I developed and applied test scenarios to validate wake-up triggers based on analog signals.
\vspace{0.25cm}

\noindent Finally, I completed a three-month internship in Iasi, Romania, during which I worked on activating the Hardware Security Manager (HSM) for a Renesas microcontroller, the RH850. I needed to understand how this module operates and carry out its initial activation, while ensuring it could still be unlocked. Indeed, in theory, once the HSM is activated, reprogramming the microcontroller is no longer possible.

\subsection{Skills Acquired}
Throughout my apprenticeship, I developed a variety of technical and professional skills:
\begin{table}[H]
    \centering
    \arrayrulecolor{black} % Defines the color of the table borders
    \renewcommand{\arraystretch}{1.5} % Adjusts the vertical spacing between rows
    \begin{tabular}{|p{3.5cm}|p{8cm}|c|}
        \hline
        \rowcolor[gray]{0.8}
        \textbf{Skill} & \textbf{Description} & \textbf{Level of Mastery} \\
        \hline
        Embedded Systems Development & Proficient in designing and implementing drivers for embedded systems (e.g., PWM driver development). & Advanced \\
        \hline
        Microcontroller Programming & Knowledge of microcontroller architectures (e.g., Infineon TC4, Renesas RH850). & Advanced \\
        \hline
        Software Validation & Experience in unit, integration, and black-box testing of embedded modules. & Advanced \\
        \hline
        Hardware Security Implementation & Hands-on experience activating and configuring Hardware Security Modules (HSM) for secure communication. & Intermediate \\
        \hline
        Low-Level Programming & Writing and debugging low-level code, including linker configuration and interrupt handling. & Advanced \\
        \hline
        Software Engineering Principles & Applied modular and scalable development practices, including BuildUnit management. & Advanced \\
        \hline
        Cross-Functional Communication & Experience conveying technical details to multidisciplinary teams (e.g., firmware, hardware, and testing groups). & Advanced \\
        \hline
        Documentation and Reporting & Created technical documentation and test reports for internal validation and project deliverables. & Advanced \\
        \hline
        International Work Experience & Gained cultural and professional exposure by working with multinational teams. & Intermediate \\
        \hline
    \end{tabular}
    \caption{Skills acquired since the start of my apprenticeship at Vitesco Technologies}
\end{table}

\subsection{Analysis and remarks}
\indent \indent Choosing an apprenticeship in my third year was a pivotal step in my journey. This experience allowed me to gain firsthand insight into the realities of industry and to better appreciate the importance of the courses taught at INSA. I was able to apply the knowledge acquired at school while developing new technical and professional skills. The projects I worked on covered a wide range of areas, from microcontroller driver development to activating a hardware security module. Furthermore, my stay in Romania, working with the Iasi teams, provided me with a rich cultural and professional immersion and allowed me to collaborate with international experts.
\vspace{0.25cm}

\noindent I found the Hardware Security Manager (HSM) activation project particularly interesting because it challenged me with secure system design and showed how the security principles studied in class are put into practice. I also learned to communicate more effectively and coordinate with geographically dispersed teams, which strengthened my project management and international collaboration skills. Getting up to speed with a new microcontroller such as the Renesas RH850 was a steep learning curve that required perseverance and teamwork to overcome.
\vspace{0.25cm}

\noindent Finally, balancing academic and professional obligations enhanced my time management skills and helped me to set clear priorities. This is why I consider my apprenticeship at Vitesco Technologies a major turning point in my engineering career: it allowed me to deepen my technical expertise, broaden my perspective on international teamwork, and confirm my interest in cybersecurity and security.


\newpage
% Cloud and Edge Computing
\section{Innovative Project - Wispers}

\subsection{Context}
\subsection{Skills acquired}

\begin{table}[h!]
    \centering
    \arrayrulecolor{black} % Defines the color of the table borders
    \renewcommand{\arraystretch}{1.5} % Adjusts the vertical spacing between rows
    \begin{tabular}{|p{11cm}|c|c|}
    \hline
    \rowcolor[gray]{0.8}
    \textbf{Skills} & \textbf{Required} & \textbf{Achieved} \\ \hline
    \rowcolor[gray]{0.9} \textbf{Manage an innovative project} &  &  \\ \hline
    Solve a problem in a creative way & 4 & 4 \\ \hline
    Develop the first stage of innovation & 4 & 4 \\ \hline
    Understand production, validation, distribution, acceptability, and aftermath of innovation & 4 & 4 \\ \hline
    Structure and lead an innovative project & 4 & 4 \\ \hline
    \end{tabular}
    \caption{Skill matrix for innovative project management and self-evaluation}
    \label{table:skills-innovative}
\end{table}

\newpage
% Cloud and Edge Computing
\section{Cloud and Edge Computing}
\subsection{Context}
%Le cours sur le Cloud and Edge Computing a été le premier cours que j'ai eu en 5ISS. Il introduit les concepts clés de l'informatique distribuée et omniprésente. Il met en lumière des architectures telles que le « Service Oriented Computing (SOC) », la virtualisation, ou encore le rôle des technologies cloud dans le développement de l'informatique omniprésente et utilitaire.
\indent \indent Among the first courses I came across during my final year at 5ISS was Cloud and Edge Computing. This course introduced me to core concepts of both distributed and pervasive computing together with the analysis of a basic architecture that included techniques of Service Oriented Computing (SOC), virtualization techniques, and cloud technologies as enablers of utility and ubiquitous computing. By nature, the course had both a theoretical and hands-on combination that gave an insight into how such paradigms shape the current IT systems and their applications.

\subsection{Technical Summary} % = synthèse du cours
%- Virtualization: Concepts, hypervisors (Type 1 and 2), paravirtualization, containerization, and VM templates. \\
%Comparison between "Service Orchestration" and "Service Choregraphy". -> Which is the best? - > It depends on the context: \\
%The orchestrator slows down the system but is easier to update.  \\
%- > Depends on the development model and on the economic model. \\
%- Cloud Computing: NIST and other definitions, core characteristics such as on-demand service, scalability, pay-per-use models, and service delivery (IaaS, PaaS, SaaS). \\
%- Edge Computing: Integration with IoT, the distribution of data control, benefits (e.g., latency reduction, data sovereignty), and challenges like connectivity and limited processing power. \\
%- Fog Computing: Bridging cloud services closer to data sources for improved real-time interaction and reduced data travel distance. "Cloudifying" the IOT world !!!

\indent \indent Virtualization has been amply covered in class. To be more specific, we discovered the different types of it: hypervisors, Type 1 and 2, paravirtualization, and containerization. We have also considered Virtual Machine Templates, which strongly ease virtual environments management and deployment. A specific highlight was the comparison between service orchestration and service choreography, where we analyzed the strengths and limitations of each. While orchestration does indeed ease updates by adopting a centralized approach, it tends to make the system slower. On the other hand, choreography is decentralized, enhancing autonomy but with a very painstaking design process. The choice between them is necessarily dependent on the specific contexts of development and economics in which they are applied.\\
\vspace{0.25cm}

Another important area discussed in this course was cloud computing: a discussion of definitions, among them the widely recognized framework from NIST, but also established its core characteristics. On-demand service provision, scalability, pay-per-use, and supply of services in models like IaaS, PaaS, and SaaS-all these form key elements that make cloud computing. We also looked at edge computing, which is integral to the IoT ecosystem, and allows for latency reduction and ensures data sovereignty. However, this also presents challenges of connectivity and the processing power of the edge devices. Finally, we discussed fog computing: an intermediary layer that connects cloud services with data sources, reducing the distance data needs to travel, and allowing real-time interaction. This was aptly referred to as the process of "cloudifying the IoT world."

\subsection{Practical work}
%La partie pratique s'est organisée en 2 TP, un sur la découverte des Cloud Hypervisors et l'autre sur l'orchestration de services dans un environnement hybride cloud/edge. Ces TP m'ont permis d'approfondir mes connaissances des concepts et des technologies utilisées dans les différentes techniques de virtualisation.
\indent \indent The practical work for this course was divided into two major sessions, each designed to reinforce the theoretical aspects covered in class. In the first session, we explored cloud hypervisors by engaging with tools such as VirtualBox and the OpenStack API. This practical session allowed me to experiment with various virtualization techniques and understand their implementation in controlled environments. The second session was on service orchestration in hybrid cloud and edge environments. In that exercise, I learned the deployment and management of a distributed system operating across these two architectures, thus providing insights into their complexities and potential.
\vspace{0.25cm}

These practical sessions were crucial for bridging the gaps in theory into applications. These have provided a chance not only to comprehend the conceptual aspects but also to apply the same in realistic scenarios. This solidifies my comprehension and confidence in handling Hybrid Architectures.

\subsection{Skills acquired}
%- Proficiency in cloud and edge computing architecture design. \\
%- Skills in using virtualization tools and managing virtual environments (VMs and containers). \\
%- Expertise in deploying services using container orchestration platforms like Kubernetes. \\
%- The ability to design resilient and low-latency systems tailored for IoT and distributed scenarios. \\
%- Insight into handling edge node variability, ensuring service continuity despite real-time disruptions.
\indent \indent The course really enhanced my technical capability and comprehension of Distributed Systems. Cloud and Edge computing architectures focusing on high performance and dependability. My experience with VirtualBox and OpenStack gave me hands-on experience in creating and managing virtual environments: virtual machines and containers. This also includes the development of expertise in deploying and orchestrating containerized services through the use of platforms like Kubernetes, which is critical in modern-day designing of distributed systems. The course enhanced my system design skills for low latency and high resilience, particularly in IoT and Distributed Environments. Apart from that, I learned real-time disruption management at the edge and continuity of service, despite variability in edge nodes.

\subsection{Analysis and remarks}
%Le format des TP est pour moi très intérressant car étant divisé en une partie théorique, permettant de fixer les concepts essentiels du cours en début de TP, accompagné d'une partie pratique, manipulation Virtual Box et l'API OpenStack, tout en permettant de mettre en application ces concepts de cours.
\indent \indent One of the strong points of this course is the structuring of the practical sessions, which did combine theory and practice quite nicely. Each session started with a theoretical introduction to reinforce the central concepts of the course; we then had to go through an implementation phase by putting these ideas into practice with tools such as VirtualBox and OpenStack APIs. In this way, both my understanding was solidly shaped and practical skills were gained; hence, learning has been both engaging and very effective. I found this format particularly enriching because it mirrors real-world workflows, where a deep understanding of concepts must complement practical expertise.

\subsection{Reflections}
\indent \indent This course gave me a better understanding of the interaction between cloud and edge computing systems and how they enable innovative, scalable, and efficient solutions. The smooth integration of theoretical and practical aspects has prepared me for the challenges of designing and managing distributed systems in professional life. In this learning process, I developed not only my technical skills but also the problem-solving approach needed to handle modern IT system complexities.

\newpage
% 5G Technologies
\section{5G technologies}
\subsection{Context}

\indent \indent The 5G technologies module presented by Professor Etienne Sicard, focused on the rapid evolution of cellular networks, emphasizing the transformative nature of 5G. A unique aspect of the learning approach was the use of reverse pedagogy, where students presented the majority of the content. The central goal was to explore cutting-edge topics in mobile communications, ranging from modulation techniques to the societal impacts of 5G and beyond.
\vspace{0.25cm}

\noindent 5G represents a paradigm shift in mobile networks, introducing Software Defined Radio (SDR) and microservices, which revolutionized network architecture by increasing flexibility and efficiency. Through this course, we students, were encouraged to examine the current state of cellular technology while considering its implications for the future.

\subsection{Technical and practical work}

\indent \indent During this course we could discuss about many varied subject. I collaborated with Noel Jumin on a presentation titled "Samsung's Vision for 6G," analyzing how Samsung's ambitions for 6G diverged from its competitors. We highlighted Samsung's focus on sub-THz frequencies, AI integration, and next-generation services such as holographic communications. This required an in-depth understanding of how 5G technologies paved the way for these future advancements, including the challenges posed by scaling up.
\vspace{0.25cm}

\noindent Other students' presentations covered topics such as 5G modulation techniques, vehicular networks, and the environmental and societal implications of 5G and 6G. These presentations provided a comprehensive understanding of both the technical and non-technical aspects of mobile communications.

\subsection{Skills acquired}
\begin{table}[h!]
    \centering
    \arrayrulecolor{black} % Defines the color of the table borders
    \renewcommand{\arraystretch}{1.5} % Adjusts the vertical spacing between rows
    \begin{tabular}{|p{3.5cm}|p{8cm}|c|}
    \hline
    \rowcolor[gray]{0.8}
    \textbf{Competence} & \textbf{Description} & \textbf{Level of Mastery} \\
    \hline
    Mobile Communications Development & Understanding the major development phases for mobile communications and development of the associated technology. & Advanced \\
    \hline
    Impact of New Mobile Technology & Understanding the impact of new mobile technology. & Intermediate \\
    \hline
    \end{tabular}
    \caption{Competences Gained During the 5G Technologies Module}
\end{table}

\subsection{Analysis and remarks}
\indent \indent This course was highly engaging as it dealt with contemporary and practical topics in cellular network technology. The discussion of 5G as a foundational step towards 6G was particularly fascinating, showing how each generation builds upon the previous one. The reverse pedagogy approach fostered a dynamic and interactive learning environment.
\vspace{0.25cm}

\noindent However, I found reverse pedagogy less effective overall. While preparing and delivering our own presentations allowed us to master our chosen topic, it made it harder to fully comprehend the topics presented by others. During the presentations, my focus was often divided between understanding the other groups' content and refining my own work.
\vspace{0.25cm}

\noindent I believe that a better balance between lectures by the teacher and student presentations would enhance the learning experience. Teacher-led lectures could provide a deeper dive into each topic, ensuring a more comprehensive understanding, while our presentations could supplement this with detailed case studies or specific insights. As it stands, I feel I gained high-level knowledge of the subjects presented by others, but lacked the depth that direct teaching could have provided.

\newpage
% Service orienté architecture
\section{Service Architecture - Software Engineering}
\subsection{Context}
3 courses : \\
The SOAP standard: In this course, you have studied the SOA architecture, the web service technology and its standards. \\
RESTFull services: \\
Microservices: How can monolithic applications can be limiting in developping flexible and scalable applications. I have studied SOA and REST architecture and what they bring to overcome the monolithic issues. Finally, I have discovered the Microservices architectures that extends the SOA architecture.
\subsection{Technical = synthèse du cours}
\subsection{Practical work = Projets de TP}
\subsection{Skills acquired}
\subsection{Analysis and remarks}

\newpage
% Wireless Sensor Network
\section{Wireless Sensor Networks}
\subsection{Context}
\indent \indent Wireless sensor networks (WSNs) are to me an essential part of the development of intelligent, interconnected systems such as the Internet of Things (IoT). These networks consist of spatially distributed sensor nodes that monitor physical or environmental conditions and relay the data wirelessly for analysis. Unlike traditional networks, WSNs prioritize energy efficiency, robustness, and scalability due to the constrained nature of their hardware and the challenging environments in which they often operate.

The course on WSNs, taught by the Professor Daniela Dragomirescu, emphasizes the importance of balancing cost, energy efficiency, lifetime, and ease of deployment when designing such systems. It introduces me to the trade-offs involved in protocol design, from the physical layer to the application one, with a focus on practical deployment strategies and performance optimization.
The second part of the course followed different approach, where students were tasked with presenting a detailed analysis of specific protocols for WSNs. My group focused on LoRa, while other groups presented on Sigfox, BLE, ZigBee, NB-IoT (5G), and M2M (5G) protocols.

\subsection{Technical}  

\indent \indent In this course I discovered the unique specifications of WSN protocols such as Zigbee and Bluetooth, and how they differ from the traditional networks.  One of the key focus was on the critical requirement for low power consumption, as many WSN nodes are restricted energy environment. One of the most common energy-saving strategies involves inactive and sleep period time.
\smallskip
In the second assignment, we explored various Medium Access Control (MAC) layer protocols, which play a vital role in WSN efficiency. We can  broadly classified them into three categories:

\begin{itemize}
    \item \textbf{Contention-based protocols:} Nodes transmit data when the communication medium is free. This approach, often implemented using Carrier Sense Multiple Access with Collision Avoidance (CSMA/CA), is simple but can lead to collisions in high-traffic scenarios.
    \item \textbf{Scheduled-based protocols:} Nodes are assigned specific time slots for transmission, ensuring collision-free communication. Time Division Multiple Access (TDMA) is a common example, providing guaranteed bandwidth but requiring precise synchronization.
    \item \textbf{Hybrid protocols:} These combine features of both contention-based and scheduled-based approaches. For example, Zigbee uses a hybrid MAC layer, with part of its superframe dedicated to contention-based access and another part reserved for scheduled transmissions.
\end{itemize}

As part of this course, I worked on a project where my group and I designed a protocol for a real-world application in a constrained environment. In a specific topic: public sewer pipes. The objective was to deploy multiple sensors along the sewer pipelines to monitor their conditions. This challenging environment required a protocol that could meet several specific constraints:

\begin{itemize}
    \item \textbf{Low energy consumption:} Essential for ensuring long-term operation.
    \item \textbf{Very low bandwidth:} To handle sparse data transmission efficiently.
    \item \textbf{Multi-hop communication:} Each sensor would act as a relay to transmit data further down the network.
\end{itemize}

For this project, we developed a new MAC protocol called 3N-MAC (Near Node Network for Wireless Sensor Networks). My role in the project focused on the implementation of the physical layer, with my teamates Paul Jaulhiac and Cyril Vasseur. We used GNU Radio to implement and test the physical layer, ensuring robust and reliable signal transmission despite the challenging environment. This hands-on experience allowed me to understand how the physical and MAC layers interact and the complexities involved in designing a system that balances energy efficiency, reliability, and performance.

\subsection{Practical work}
\subsection{Skills acquired}
Through this course, its research part, and the labs, I consider to have acquired the following skills:
\begin{itemize}
    \item \textbf{Protocol Analysis and Design:} Understanding and evaluating the trade-offs between different WSN MAC protocols to design networks tailored to specific application requirements (Being able to suggest optimal technological solutions for  IoT networks).
    \item \textbf{Simulation and Evaluation:} Using simulation tools to model and assess network performance under diverse environmental and operational constraints.
    \item \textbf{Energy Optimization:} Developing strategies to balance energy efficiency and communication reliability in WSNs.
    \item \textbf{Critical Comparison:} Comparing and selecting the most suitable protocols for various IoT applications based on factors such as energy efficiency, scalability, and reliability (Being able to analyze and evaluate optimal wireless network technologies).
    \item \textbf{Problem Solving in Constrained Environments:} Designing practical solutions for challenging real-world scenarios, such as our sewer pipe monitoring project.
\end{itemize}

\subsection{Analysis and remarks}

\indent \indent Throughout this course, I realized how critical it is to select the right protocol when designing a Wireless Sensor Network (WSN). Each protocol has its own strengths and trade-offs, making it essential to align the choice with the specific requirements of the application.

For example, protocols like S-MAC and T-MAC are excellent for conserving energy, but their reliance on synchronization and inability to handle mobility well make them better suited for static networks with moderate traffic. On the other hand, Z-MAC stood out for its hybrid approach, which balances energy efficiency and throughput by adapting to traffic levels. However, I noticed that Z-MAC’s reliance on global synchronization can be a challenge in highly dynamic environments.

Working on our project also helped me understand the practical challenges of deploying WSNs in constrained environments. For our sewer pipe monitoring system, we needed to balance low energy consumption, reliability, and scalability. While designing the 3N-MAC protocol, I gained hands-on experience in tailoring a solution to meet the specific demands of the application. However, one major difficulty was the implementation of the physical layer using GNU Radio. Although we had a brief introduction to the software, it was not sufficient to fully understand how to implement a working physical layer. For future projects, I think it would be extremely useful to have more detailed guidance or tutorials on how to use tools like GNU Radio to implement protocols effectively.
\\
\\
\indent Another takeaway from this course is that there is no one-size-fits-all solution for WSNs. For instance, energy-efficient protocols like B-MAC are ideal for low-traffic applications but can introduce delays or lack features like synchronization. Similarly, location-based protocols like GAF are great for dynamic networks but require accurate localization, which can complicate the deployment.

Overall, this course has deepened my understanding of the complexities involved in WSN design and deployment. It highlighted the importance of making informed trade-offs based on the specific needs of the application and provided me with a strong foundation to approach similar challenges in the future. The hands-on project reinforced the theoretical knowledge, but with more practical training on software tools, I believe I could have gained even more from the experience.

\newpage
% Middleware for IOT
% \section{Middleware for IoT}
% \subsection{Context}
% Le cours de Middleware for IoT a été donné par le professeur Thierry Monteil. Les objectifs principaux de ce cours sont de découvrir différentes approches et protocoles de communication, et de comprendre les enjeux de la communication entre objets connectés afin d'être capable de mettre en place des solutions adaptées au besoin.\\
% Ce cours, comme pour le cours de \hyperref[sec:service_architecture]{Service Architecture} donné par Ms. Nawal Guermouche, était sous la forme d'un MOOC (Massive Open Online Course), à la suite duquel nous avons eu une serie de travaux pratiques. Ces TP qui visent a illustrer les concepts abordés dans le MOOC, notamment l'utilisation d'MQTT, oneM2M et de Node-RED.\\
% L'objectif des TP était de comprendre comment faire intéragir plusieurs objets connectés entre eux, en choisisant les protocoles adaptés, tout en prenant en compte la diversité des infrastructures matérielles et logicielles exixtantes.

% \subsection{Technical Summary}
% Comme dit précédemment, durant ce cours nous avons abordés plusieurs protocoles de communication, et notamment Message Queuing Telemetry Transport (MQTT). Il s'agit d'un protocole de messagerie léger, conçu pour les appareils ayant des ressources limitées. Son fonctionnement est basé sur un broker, un serveur, qui gère des sujets de publication (des topics). Les clients "publishers" publient des messages sur des topics, et les clients "subscribers" recoivent des messages en s'abonnant à des topics.\\
% Grace à son architecture simple, le protocole MQTT est très utilisé dans l'IoT, où la consommation de bande passante doit être minimisée, et la fiabilité y être essentielle.\\

% Le second protocole que nous avons étudié est oneM2M. Il s'agit plus précisemment d'une norme soutenue par plusieurs organismes de standardisation, qui vise à fournir et assurer l'intéropérabilité des architectures IoT. Basée sur REST, oneM2M propose une structure hiérarchique permettant de modéliser des ressources pour une application IoT. Les ressources sont organisées en trois niveaux : les Application Entities (AE), les Containers (CNT) et les Content Instances (CI). Ce sont ces niveaux définissent l'espace de stockage hiérarchique des Servers. Les Application Entities (AE) représentent les fonctionnalités disponibles dans lesquelles il y a des containers (CNT) qui sont des catégories de données et des Content Instance (CI) qui sont des instances de données publiée.\\
% Ainsi, c'est au travers d'un container dans une AE, qu'un appareil IoT peut publier ou lire des informations.\\

% \subsection{Practical Work}
% Le premier projet des séances de TP, a été le déploiemment d'un broker MQTT, avec le logiciel Mosquitto, et la mise en place d'échanges de messages simple entre plusieurs ESP8266. Un ESP8266 jouant le role du bouton, le publisher, et un autre controllant une LED, le subscriber. Chaque changement de l'état du bouton était reflété sur l'extinction ou l'allumage de la LED.\\

% Le deuxième projet, qui était sensiblement le même que le premier mais avec oneM2M, a commencé par le suivit d'un tuto sous un Jupyter Notebook, pour en comprendre les concepts. Les premiers exercices de ce notebook nous ont permis de comprendre comment créer des AE, des containers, et des content instances. Puis, nous sommes allé plus loin en découvrant les créations de groupes, les notifications et les règles qui permettenht de mettre a jour les ressources de manière automatique.\\
% Ce n'est qu'une fois ces concepts  maîtrisés que nous avons utilisé un serveur oneM2M et un script Python pour simuler la même application qu'avec MQTT. Afin d'avoir quelque chose d'équivalent au premier projet, mon binome et moi avons simulé un “bouton” et une “LED” virtuelle. Un script Python nous permettait de créer des AE, des containers, et des content instances pour le bouton et la LED, et de gérer les notifications. 

% Si nous avions eu plus de temps sur ce projet, j'aurais aimé pouvoir intégrer Node-RED pour pouvoir manager plus facilement les flux de données entre les différents appareils IoT. Node-RED nous aurait apporté une interface visuelle, connectant les dispositifs matériels, les API et des services en ligne de manière intuitive.

% \subsection{Skills Acquired}
% \begin{table}[H]
%     \centering
%     \begin{tabular}{|p{3.5cm}|p{8cm}|p{3.5cm}|}
%     \hline
%     \textbf{Skill} & \textbf{Description} & \textbf{Level of Mastery} \\
%     \hline
%     Situate IoT Standards & Know how to situate the main standards for the Internet of Things & Advanced \\
%     \hline
%     Deploy IoT Architecture & Deploy an architecture compliant to an IoT standard and implement a sensor network & Intermediate \\
%     \hline
%     Configure OM2M & Deploy and configure an IoT architecture using OM2M & Intermediate \\
%     \hline
%     Interact with REST & Interact with the different resources of the architecture using REST services & Advanced \\
%     \hline
%     Integrate New Technology & Integrate a new technology into the deployed architecture & Intermediate \\
%     \hline
%     \end{tabular}
%     \caption{Skills Acquired}
%     \label{tab:skills_acquired}
% \end{table}


% \subsection{Analysis and remarks}

% La découverte et l'expérimentation de deux protocoles IoT complémentaires, MQTT (léger, adapté à des cas d'usage rapide et peu gourmands en ressources) et oneM2M (plus structuré et standardisé, offrant un cadre complet pour des applications IoT à grande échelle), m'ont permis de mieux saisir l'importance de l'interopérabilité dans l'Internet des Objets. La mise en pratique a été particulièrement instructive, de la configuration d'un broker Mosquitto, à la conception d'un mini-projet sur des ESP8266 et l'exploration de oneM2M au travers de scripts Python, en passant par Node-RED pour la mise en flux des données.
% Toutefois, j'aurais aimé approfondir davantage l'intégration directe de oneM2M dans Node-RED pour mieux en exploiter tout le potentiel, notamment en termes de notifications et de règles de gestion.

% Par ailleurs, l'installation des outils sur les machines personnelles nous a freinés, mon binome et moi, notamment sur la partie tuto oneM2M sur le notebook Jupyter. J'aurai aimé qu'une solution fonctionnelle, telle qu'une machine virtuelle à déployer, nous soit fourni en début de TP comme cela est fait notamment dans les TP du \hyperref[sec:service_architecture]{Service Architecture}.

% Sur le plan personnel, cette matière vient enrichir mon parcours en me donnant une vue plus globale des différentes approches de communication dans l'IoT. Les compétences acquises me seront utiles pour de futurs projets où il faudra choisir la solution la plus adaptée à un contexte précis.

% Souhaitant m'orienter vers le Master Spécialisé en TLS-Sec à l'issue de mon diplôme INSA, je regrette que la dimension sécurité de ces protocoles n'aient pas été abordé durant les travaux pratiques, car la sécurisation des échanges (chiffrement, authentification, protection contre les intrusions) constitue un enjeu majeur en IoT.

\section{Middleware for IoT}
\subsection{Context}
The Middleware for IoT course was taught by Professor Thierry Monteil. The main objectives of this course were to explore different approaches and communication protocols, and to understand the challenges of communication between connected devices in order to implement solutions tailored to specific needs.\\

Like the course \hyperref[sec:service_architecture]{Service Architecture} taught by Ms.\ Nawal Guermouche, this course was delivered in the form of a MOOC (Massive Open Online Course). Afterwards, we carried out a series of practical exercises designed to illustrate the concepts discussed in the MOOC, notably the use of MQTT, oneM2M, and Node-RED.\\

The goal of these practical sessions was to understand how multiple connected devices can interact with each other by choosing appropriate protocols, while taking into account the variety of existing hardware and software infrastructures.

\subsection{Technical Summary}
As mentioned earlier, we studied several communication protocols during this course, particularly Message Queuing Telemetry Transport (MQTT). This is a lightweight messaging protocol designed for devices with limited resources. Its functioning relies on a broker (a server) that manages publication topics. Client publishers publish messages on specific topics, and client subscribers receive messages by subscribing to these topics.\\

Thanks to its simple architecture, the MQTT protocol is widely used in IoT, where bandwidth consumption must be minimized and reliability is essential.\\

The second protocol we studied is oneM2M. More precisely, it is a standard endorsed by multiple standardization bodies, aiming to provide and ensure interoperability among IoT architectures. Based on REST, oneM2M proposes a hierarchical structure to model resources for an IoT application. The resources are organized on three levels: Application Entities (AE), Containers (CNT), and Content Instances (CI). These levels define the hierarchical storage space on servers. Application Entities (AE) represent available functionalities, within which there are Containers (CNT) that serve as data categories and Content Instances (CI) that hold published data instances.\\

Thus, it is through a container in an AE that an IoT device can publish or read information.

\subsection{Practical Work}
The first project in the practical sessions involved deploying an MQTT broker using Mosquitto and setting up simple message exchanges among several ESP8266 devices. One ESP8266 acted as a button (the publisher), while another controlled an LED (the subscriber). Any change in the button’s state was reflected by turning the LED on or off.\\

The second project, which was essentially the same as the first but with oneM2M, began with following a tutorial in a Jupyter Notebook to understand the core concepts. The first exercises in the notebook enabled us to learn how to create AEs, containers, and content instances. We then went a step further by exploring group creation, notifications, and rules that can automatically update resources.\\

Only after mastering these concepts did we use a oneM2M server and a Python script to simulate the same application we had built with MQTT. To achieve an equivalent setup to the first project, my partner and I simulated a virtual “button” and a virtual “LED.” A Python script allowed us to create AEs, containers, and content instances for both the button and the LED, as well as manage notifications. 

Had we had more time for this project, I would have liked to integrate Node-RED to more easily manage data flows among the various IoT devices. Node-RED would have provided a visual interface to intuitively connect hardware devices, APIs, and online services.

\subsection{Skills Acquired}
\begin{table}[H]
    \centering
    \begin{tabular}{|p{3.5cm}|p{8cm}|p{3.5cm}|}
    \hline
    \textbf{Skill} & \textbf{Description} & \textbf{Level of Mastery} \\
    \hline
    Situate IoT Standards & Know how to position the main standards for the Internet of Things & Advanced \\
    \hline
    Deploy IoT Architecture & Deploy an architecture compliant with an IoT standard and implement a sensor network & Intermediate \\
    \hline
    Configure OM2M & Deploy and configure an IoT architecture using OM2M & Intermediate \\
    \hline
    Interact with REST & Interact with different resources of the architecture using REST services & Advanced \\
    \hline
    Integrate New Technology & Integrate a new technology into the deployed architecture & Intermediate \\
    \hline
    \end{tabular}
    \caption{Skills Acquired}
    \label{tab:skills_acquired}
\end{table}

\subsection{Analysis and Remarks}
Discovering and experimenting with two complementary IoT protocols—MQTT (lightweight, suitable for quick use cases with minimal resource consumption) and oneM2M (more structured and standardized, offering a comprehensive framework for large-scale IoT applications)—has helped me understand the importance of interoperability in the Internet of Things. The practical work was especially instructive, from configuring a Mosquitto broker and designing a mini-project on ESP8266 devices, to exploring oneM2M via Python scripts and using Node-RED for data flow management.

However, I would have liked to delve deeper into the direct integration of oneM2M into Node-RED to fully leverage its potential, particularly in terms of notifications and management rules.

In addition, installing the necessary tools on our personal machines slowed down my partner and me, especially for the oneM2M tutorial in the Jupyter Notebook. I would have appreciated having a ready-to-use solution—such as a virtual machine—to deploy at the start of the practical sessions, similar to what was provided for the \hyperref[sec:service_architecture]{Service Architecture} TP sessions.

From a personal standpoint, this course enhanced my academic path by giving me a more comprehensive view of different communication approaches in IoT. The skills I acquired will be valuable for future projects that require choosing the most appropriate solution for a specific context.

Intending to pursue the Master’s Program in TLS-Sec after completing my INSA degree, I regret that the security aspects of these protocols were not addressed during the practical work. Ensuring secure exchanges (encryption, authentication, intrusion protection) is a major challenge in IoT.


\newpage
% Security for Connected Objects
\section{Security for connected objects}
\subsection{Context}
\subsection{Technical}
\subsection{Practical work}
\subsection{Skills acquired}
\subsection{Analysis and remarks}

\newpage
% Energy for IOT (or Connected Objects)
\section{Energy for connected objects}
\subsection{Context}
\paragraph{}The "Energies for Connected Objects" course dealt with pioneering strategies to feed IoT devices without batteries or wired systems. This training focused on Ambient Energy Harvesting and Wireless Power Transfer applied to the high demand for energy-autonomous solutions in healthcare, smart cities, and environmental monitoring, among others. It brought together a multidisciplinary approach, from theoretical insights to efficient, sustainable, and reliable energy system design.

\subsection{Technical}
\paragraph{}It contained discussions on capacitors and supercapacators to buffer energy, wireless power transfer that has been highly evolved with a near-field mechanism by means of capacitive and inductive coupling to a far-field radiative by the use of a rectenna, and energy harvesting techniques using ambient energy through conversion of energy due to light, mechanical motion, thermal gradient, and electromagnetic waves. The course also integrated optimization strategies, such as antenna design to match the frequency and software optimizations that would reduce energy consumption in connected devices.

\subsection{Practical work}
\paragraph{}The practical work for this course involved the design and testing of energy systems in the laboratory. One of the projects was to power an LED from ambient RF energy. This involved analyzing a rectifier circuit, choosing appropriate antennas, and optimizing the energy transfer for efficient operation. I used GNURadio and spectrum analyzers to measure RF power output, find the optimal operating frequencies, and test various energy storage configurations. The "store then use" strategy, as realized with a bq25504 power management unit and a TPS63031 DC-DC converter, allowed for efficient energy buffering and utilization in low-power scenarios.

\subsection{Skills acquired}
\paragraph{}In this course, I developed high-level technical expertise in low-power circuit design, energy harvesting systems, and wireless power transfer technologies. I acquired hands-on experience with laboratory tools and methodologies such as frequency sweeps, impedance matching, and antenna characterization. Beyond the technical skills, I learned to interpret complex system data and optimize designs for practical applications. The course also fostered innovative problem-solving skills, particularly in balancing theoretical frameworks with real-world constraints.

\subsection{Analysis and remarks}
\paragraph{}This class demonstrated the potentials and pitfalls of designing energy-autonomous IoT devices. Even though I implemented various systems, such as the energy harvesting LED with ease, environmental inputs had high variability that posed the biggest challenges. For instance, supplying power continuously in constantly fluctuating electromagnetic environments forced me to devise creative solutions but also highlighted the necessity of using hybrid energy systems. Furthermore, a central design challenge was determining the balance between optimizing the capture of energy and reliability of the system.

\subsection{Reflections}
\paragraph{}"Energy for Connected Objects" provided a great platform for innovative energy solution development within the IoT domain. It identified the key challenges arising from future technologies in relevance to sustainability and efficiency; therefore, this was focused on the latest energy system approaches and applications. In the future, my skills and knowledge will guide my work in energy-autonomous devices, mainly in scalable wireless sensor networks, but also including advanced material development for energy conversion and storage. These experiences crystallized my aspiration to further the frontier of IoT energy systems toward sustainable technological development.

\newpage
% LAB at AIME
% \section{LAB at AIME}
% \subsection{Context}
% \indent \indent L'INSA Toulouse possède un Atelier Interdisciplinaire de Microélectronique et d'Électronique (AIME) qui est un laboratoire de recherche et de formation en microélectronique et électronique. Dans le cadre de la formation ISS, nous avons passé y avons passé une semaine pour réaliser un capteur de gaz par synthèse chimique de nanoparticules $WO_3$ et leur intégration sur des dispositifs de micro-électronique. Ce laboratoire est équipé de salles blanches permettant d'effectuer des manipulations de haute précision nécéssaire pour la fabrication à la micro-echelle.

% Cependant, les étapes initiales du processus de fabrication, jusqu'à la photolithographie numéro 2 (« Contacts Opening »), avaient déjà été réalisées par les techniciens du laboratoire pour nous permettre de nous concentrer sur les étapes suivantes.

% \subsection{Technical Summary}
% Le projet visait à produire un capteur de gaz basé sur des nanoparticules de $WO_3$, un matériau semi-conducteur connu pour ses propriétés sensibles aux gaz. Les étapes principales étaient dans l'ordre de : 
% \begin{enumerate}
%     \item Réaliser la photolithographie numéro 3 pour le gravage des électrodes métalliques en aluminium.
%     \item Assembler des dispositifs, incluant le découpage et le montage des puces sur des supports TO5.
%     \item Synthétiser des nanoparticules de $WO_3$, comprenant la préparation des graines et la croissance de nanofils par des procédés hydrothermaux.
%     \item Intégrer des nano bâtonnets de $WO_3$ sur les peignes interdigités via la diélectrophorèse.
%     \item Faire caractérisation électrique des capteurs dans une atmosphère contrôlée, contenant de l'éthanol ou de l'ammoniaque.
% \end{enumerate}

% \subsection{Practical work}
% Au cours de la semaine nous avons pu réaliser les étapes 1 à 5 du TP. Nous travaillions en groupe de 4, et chaque groupe était responsable de son capteur. J'ai été en charge de la synthèse des nanoparticules de $WO_3$ et de leur intégration sur les peignes interdigités. J'ai également participé à la caractérisation électrique des capteurs.

% \subsection{Skills acquired}
% \indent \indent Grâce à ce laboratoire, je considère pense avoir acquis les compétences suivantes :
% \begin{table}[H]
%     \centering
%     \begin{tabular}{|p{5cm}|p{8cm}|p{3cm}|}
%     \hline
%     \textbf{Compétence} & \textbf{Description} & \textbf{Niveau de maîtrise} \\
%     \hline
%     Compréhension des notions de base des capteurs et de l'acquisition de données & Comprendre les principes fondamentaux des capteurs, de la physique, de l'électronique et de la métrologie pour l'acquisition de données. & Avancé \\
%     \hline
%     Fabrication de capteurs à base de nanoparticules & Être capable de réaliser un capteur de gaz en utilisant des outils de microélectronique : synthèse chimique, intégration et tests. & Intermédiaire \\
%     \hline
%     Utilisation des outils de microélectronique en salle blanche & Respecter les protocoles et manipuler des équipements pour des étapes telles que la photolithographie, la métallisation et l'intégration des nanoparticules. & Avancé \\
%     \hline
%     Etre capable de réaliser une datasheet & Savoir rédiger une datasheet pour un capteur de gaz réalisé, incluant les caractéristiques électriques, et les conditions de test. & Intermédiaire \\
%     \hline
%     \end{tabular}
%     \caption{Compétences acquises lors de la semaine au laboratoire AIME}
% \end{table}

% Ne venant pas d'une formation de physicien ou de chimiste, je ne peux pas dire que je maîtrise complètement le sujet. Mais je peux dire qu'après cette semaine au laboratoire, je serais capable de suivre un protocole de fabrication, et de caractériser un capteur.

% \subsection{Analysis and remarks}
% \indent \indent Etant en alternance, cette semaine à l'AIME a été pour moi une vrai découverte du monde de la recherche et plus particulièrement de la microelectronique. J'ai pu découvrir et manipuler sur des étapes de fabrication de dispositifs microélectroniques, allant de la photolithographie à l'intégration de nanoparticules.

% L'organisation en groupe de 4 m'a également permis d'appréhender le travail en équipe dans un context de recherche, en partageant les tâches et en collaborant pour atteindre un objectif commun. Je pense que cette expérience m'a permis de mieux comprendre les enjeux et les défis de la recherche, et des challenges que sont de devoir suivre un protocole expérimental rigoureux, sans se tromper dans les manipulations.

\section{LAB at AIME}

\subsection{Context}

\indent \indent INSA Toulouse has an Interdisciplinary Microelectronics and Electronics Workshop (AIME), which is a research and training laboratory in microelectronics and electronics. As part of the ISS program, we spent a week there to build a gas sensor through the chemical synthesis of $WO_3$ nanoparticles and their integration into microelectronic devices. This lab is equipped with clean rooms that allow high-precision procedures, essential for micro-scale fabrication.

However, the initial fabrication steps, up to photolithography step 2 (“Contacts Opening”), had already been carried out by the lab technicians, allowing us to focus on the subsequent stages.

\subsection{Technical Summary}

The project aimed to produce a gas sensor based on $WO_3$ nanoparticles, a semiconductor material known for its gas sensitivity. The main steps were, in order:

\begin{enumerate}
    \item Carry out photolithography step 3 for the aluminum electrode etching.
    \item Assemble the devices, including cutting and mounting the chips on TO5 supports.
    \item Synthesize $WO_3$ nanoparticles, involving seed preparation and nanowire growth via hydrothermal processes.
    \item Integrate $WO_3$ nanorods onto interdigitated combs using dielectrophoresis.
    \item Perform electrical characterization of the sensors in a controlled atmosphere containing ethanol or ammonia.
\end{enumerate}

\subsection{Practical Work}

During the week, we were able to complete steps 1 through 5 of the lab. We worked in groups of four, and each group was responsible for its own sensor. I was in charge of synthesizing the $WO_3$ nanoparticles and integrating them onto the interdigitated combs. I also participated in the electrical characterization of the sensors.

\subsection{Skills Acquired}

\indent \indent Thanks to this lab, I believe I have acquired the following skills:

\begin{table}[H]
    \centering
    \arrayrulecolor{black} % Defines the color of the table borders
    \renewcommand{\arraystretch}{1.5} % Adjusts the vertical spacing between rows
    \begin{tabular}{|p{3.5cm}|p{8cm}|c|}
    \hline
    \rowcolor[gray]{0.8}
    \textbf{Skill} & \textbf{Description} & \textbf{Level of Mastery} \\
    \hline
    Understanding Basic Concepts of Sensors and Data Acquisition & 
    Understanding the fundamental principles of sensors, physics, electronics, and metrology for data acquisition. & 
    Advanced \\
    \hline
    Fabrication of Nanoparticle-based Sensors & 
    Being able to build a gas sensor using microelectronics tools: chemical synthesis, integration, and testing. & 
    Intermediate \\
    \hline
    Using Microelectronics Tools in Clean Rooms & 
    Following protocols and handling equipment for steps such as photolithography, metallization, and nanoparticle integration. & 
    Advanced \\
    \hline
    Creating a Datasheet & 
    Knowing how to write a technical datasheet for a gas sensor, including electrical characteristics and test conditions. & 
    Intermediate \\
    \hline
    \end{tabular}
    \caption{Skills acquired during the week at the AIME laboratory}
\end{table}

Not coming from a physicist or chemist background, I can't say I've completely mastered the subject. However, after this week in the lab, I feel capable of following a fabrication protocol and characterizing a sensor.

\subsection{Analysis and remarks}

\indent \indent As a work-study student, this week at AIME was a real discovery for me of the research environment, and more specifically of microelectronics. I was able to explore and handle various steps in the fabrication of microelectronic devices, ranging from photolithography to nanoparticle integration.
\vspace{0.25cm}

\noindent Working in groups of four also allowed me to understand teamwork in a research context, by sharing tasks and collaborating to achieve a common goal. I believe this experience has helped me to better grasp the challenges of research as well as the importance of following a strict experimental protocol without making errors during the procedures.


\newpage
% Microcontrollers Open-Source Hardware and Sensor Introduction
% \section{Microcontrollers Open-Source Hardware and Sensors Introduction} 

% \subsection{Context}
% %Après la semaine à l'AIME pour la réalisation des capteurs de gaz, nous avons eu des cours au département de physique.
% %Ces cours étaient axés sur la réalisation de la datasheet du capteur et le dimensionnement des composants pour l'étage d'adaptation permettant lire les valeurs du capteur.
% %Nous avons aussi eu une introduction sur les microcontrôleurs, sur l'utilisation de KiCad pour la réalisation de PCB, sur le fonctionnement de chirspack et de NodeRed.
% %L'objectif de ce module était de nous apprendre à réaliser un système complet allant du capteur de gaz, en passant par la datasheet du capteur, jusqu'à une interface web utilisateur sous NodeRed.

% \indent \indent Après une semaine à l'AIME dédiée à la réalisation de capteurs de gaz, nous avons suivi des cours au département de physique. Ces cours portaient sur la création de la fiche technique du capteur et le dimensionnement des composants pour l'étage d'adaptation permettant de lire les valeurs du capteur. Nous avons également eu une introduction aux microcontrôleurs, à l'utilisation de KiCad pour la conception de PCB, ainsi qu'au fonctionnement de Chirspack et de NodeRed. L'objectif de ce module était de nous apprendre à réaliser un système complet, allant du capteur de gaz à la fiche technique du capteur, jusqu'à une interface utilisateur web sous NodeRed.

% \subsection{Technical Summary}
% %1- Semaine AIME
% %2- Microcontroller introduction (LoRa et Chirspack/TTN)
% %3- Dimensionnement LTSPICE
% %4- Git introduction
% %5- App MIT inventor
% %6- NodeRed introduction
% %7- KiCad introduction
% %8- Modélisation/Realisation de la PCB

% \indent \indent Ce module, qui s'est étalé sur plusieurs semaines, a commencé par la semaine de réalisation des capteurs de gaz à l'AIME. Ensuite, nous avons eu une introduction aux microcontroleurs, plus à l'attention des étudiants venant de la filière IR, qui ont pu découvrir les microcontroleurs en manipulant des cartes Arduinos.
% Etant plus à l'aise, car provenant de la filière AE, mon binome et moi avons choisit de travailler sur un microcontrolleur ESP8266, et de le connecter au réseau LoRa de l'INSA à l'aide d'un module RN2483.

% Ce module était cependant très peu fonctionnel, probablement à cause des erreurs d'alimentation faites par les étudiants des précédentes années qui l'interfacait avec un arduino. Ce module est sensé être alimenté en 3.3V, mais les signaux des arduinos sont en 5V, ce qui sans étage d'adaptation de tension endommage le module. Face a ces problèmes, j'ai décidé d'utiliser mes modules esp32 personnels, qui ont un module LoRa intégré. Nous avons aussi rencontré des problèmes avec la gateway chirspack de l'INSA, qui étaient souvent saturée, et qui nous permettait d'envoyer des trames 1 fois sur 10. J'ai donc décidé de travailler avec ma propre gateway, qui est elle enregistrée sur The Thing Network (TTN), et qui fonctionnait parfaitement. Une fois les trames LoRa visualisable sur le site de TTN, nous avons utilisé un capteur MQ-3B pour simuler le fonctionnement de notre propre capteur de gaz.

% Nous avons ensuite eu un cours de dimensionnement de composant sur le logiciel LTSPICE, qui nous a permis de comprendre le fonctionnement de notre étage d'adaptation, et de dimensionner les composants pour qu'il fonctionne correctement.

% Nous avons aussi eu une introduction à Git, ce qui n'était pas nouveau pour moi, car j'utilise Git dans mon entreprise, mais qui a permis à mes camarades de découvrir cet outil de gestion de version.

% Le cours suivant était une introduction à l'application MIT inventor, qui permet de générer rapidement une application android. En ajoutant un module Bluetooth à notre ESP32, nous avons pu controller l'allumage d'une LED depuis l'application.

% Nous avons ensuite eu une introduction à NodeRed, qui est un outil de programmation par blocs, qui permet de créer des applications web de manière très simple. Nous avons utilisé cet outil pour créer une interface web permettant de visualiser les données de notre capteur de gaz, via un protocole MQTT pour récupérer les données de la plateforme TTN.

% Enfin, nous avons eu une introduction à KiCad, qui est un logiciel de conception de PCB. Nous avons utilisé ce logiciel pour réaliser un "shield" pour notre ESP32, qui permet de connecter, et de réaliser l'étage d'adaptation pour notre capteur de gaz. Comme nous manquions de temps pour réaliser le PCB, je l'ai commandé chez JLC PCB, et j'ai réalisé le montage des composants à sa reception.

% Vous pouvez retrouver l'intégralité du projet sur mon github: \url{https://github.com/Raspeur/AIME_sensor}

% \subsection{Skills acquired}
% \begin{table}[H]
%     \centering
%     \begin{tabular}{|p{3.5cm}|p{8cm}|p{3.5cm}|}
%     \hline
%     \textbf{Competence} & \textbf{Description} & \textbf{Level of Mastery} \\
%     \hline
%     Understand Microcontroller Architecture & 
%     Comprehending microcontroller architectures, functionalities, and usage scenarios. & 
%     Advanced \\
%     \hline
%     Data Acquisition System Design & 
%     Designing end-to-end data acquisition systems by selecting sensors, conditioning signals, and integrating microcontrollers for specific applications. & 
%     Advanced \\
%     \hline
%     Sensor Signal Conditioner Circuit & 
%     Developing and simulating signal conditioning circuits for sensors to ensure accurate data acquisition. & 
%     Advanced \\
%     \hline
%     Shield Design for Gas Sensor & 
%     Designing and fabricating custom hardware expansions (shields) to properly integrate gas sensors. & 
%     Advanced \\
%     \hline
%     Sensor Software and HMI Development & 
%     Implementing embedded software solutions, including sensor drivers, data processing, and user interfaces for gas sensing applications. & 
%     Advanced \\
%     \hline
%     Smart Device Integration & 
%     Combining hardware and software components into a cohesive, fully operational IoT device. & 
%     Advanced \\
%     \hline
%     \end{tabular}
%     \caption{Competences Gained During the Microcontrollers and Open Source Hardware Module}
% \end{table}

% \subsection{Analysis and remarks}

% \indent \indent Ce cours était pour moi très intéressant, car il m'a permis de découvrir de découvrir de nouveaux outils tels que KiCad et NodeRed, et de les utiliser dans un proket allant de bout en bout. On est parti de la réalisation d'un capteur, jusqu'à une interface web pour visualiser les données en passant par la réalisation d'un PCB. C'était très gratifiant de voir le projet se concrétiser, et de devoir gérer son temps de travail pour arriver à tout réaliser dans les temps.

% J'ai aussi apprécié de pouvoir aider mes camarades de la filière IR, qui n'avaient pas de connaissances en microcontroleurs, et qui n'étaient pas forcément très à l'aise.

% Dans le cadre de ce cours, j'ai aussi pu découvrir la notion de licenses, open-source ou propriétaires, et toutes leurs implications. C'est une notion que je considère très importante, et que je n'avais pas encore eu l'occasion d'aborder dans le cadre de mes études.

% J'ai aussi eu l'occasion de découvrir le fonctionnement de la gateway chirspack de l'INSA, et de ses limites, ce qui m'a poussé à utiliser ma propre gateway LoRa, elle connecté au LoRaWAN TTN.

\section{Microcontrollers, Open-Source Hardware, and Sensor Introduction}

\subsection{Context}

\indent \indent After a week at AIME dedicated to building gas sensors, we attended classes in the physics department. These classes focused on creating the sensor's technical datasheet and sizing the components for the adaptation stage that reads the sensor values. We also received an introduction to microcontrollers, using KiCad for PCB design, and the operation of ChirpStack and NodeRed. The goal of this module was to teach us how to build a complete system from the gas sensor and its datasheet to a web user interface in NodeRed.

\subsection{Technical Summary}

\indent \indent Spanning several weeks, this module began with the gas sensor development week at AIME. Afterwards, we had an introduction to microcontrollers, primarily aimed at students from the IR (Computer Engineering) track, who discovered microcontrollers using Arduino boards.

Since my teammate and I come from the AE speciality and were more comfortable with microcontrollers, we chose to work with an ESP8266 and connect it to INSA's LoRa network using an RN2483 module. However, this module proved to be barely functional, likely due to power supply errors made by students in previous years. Indeed, it is supposed to be powered at 3.3 V, but Arduino signals are 5 V, which damages the module in the absence of a level-shifting stage.

Faced with these issues, I decided to use my own ESP32 modules, which integrate a LoRa module directly. We also encountered problems with INSA's ChirpStack gateway, which was often saturated, allowing us to send frames successfully only once every ten attempts. I therefore decided to use my own gateway, which is registered on The Things Network (TTN) and worked perfectly. Once the LoRa frames were visible on the TTN website, we used an MQ-3B sensor to simulate our gas sensor's operation.

Next, we took a component-sizing course using LTSPICE, to understand our adaptation stage and correctly dimension the components.

We also had an introduction to Git. This was not new to me, since I use Git in my company, but it allowed my classmates to discover this version-control tool.

We then learned about MIT App Inventor, a tool for quickly building Android apps. By adding a Bluetooth module to our ESP32, we were able to control an LED from the app.

We then moved on to NodeRed, a block-based programming tool that makes it very easy to create web applications. We used it to build a web interface for viewing our gas-sensor data, retrieving these data via MQTT from the TTN platform.

Finally, we were introduced to KiCad, a PCB design software. We used it to create a “shield” for our ESP32, incorporating the adaptation stage for the gas sensor. Due to time constraints, I ordered the PCB from JLCPCB and assembled the components upon its arrival.

You can find the entire project on my GitHub: \url{https://github.com/Raspeur/AIME_sensor}

\subsection{Skills Acquired}

\begin{table}[h!]
    \centering
    \arrayrulecolor{black} % Defines the color of the table borders
    \renewcommand{\arraystretch}{1.5} % Adjusts the vertical spacing between rows
    \begin{tabular}{|p{11cm}|c|c|}
    \hline
    \rowcolor[gray]{0.8}
    \textbf{Skills} & \textbf{Required} & \textbf{Achieved} \\ \hline
    \rowcolor[gray]{0.9} \textbf{Introduction to Sensors} &  &  \\ \hline
    Understand basic notions of sensors, data acquisition: physics, electronics and metrology point of view & 4 & 4 \\ \hline
    Be able to manufacture a nano-particles sensor using micro-electronics tools: chemical synthesis, assembly, testing & 4 & 4 \\ \hline
    Be able to design the datasheet of the sensor manufactured & 4 & 4 \\ \hline
    \end{tabular}
    \caption{Skill matrix for Introduction to Sensors}
    \label{table:skills-intro-sensors}
\end{table}

\begin{table}[h!]
    \centering
    \arrayrulecolor{black} % Defines the color of the table borders
    \renewcommand{\arraystretch}{1.5} % Adjusts the vertical spacing between rows
    \begin{tabular}{|p{11cm}|c|c|}
    \hline
    \rowcolor[gray]{0.8}
    \textbf{Skills} & \textbf{Required} & \textbf{Achieved} \\ \hline
    \rowcolor[gray]{0.9} \textbf{Microcontrollers and Open Source Hardware} &  &  \\ \hline
    Understand microcontroller architecture and how to use them & 4 & 4 \\ \hline
    Be able to design data acquisition system (sensor, conditioner, microcontroller) with respect to the application & 4 & 4 \\ \hline
    Be able to design the electronic circuit of a sensor’s signal conditioner (design + simulation) & 4 & 4 \\ \hline
    Be able to design a shield to accommodate the gas sensor & 4 & 4 \\ \hline
    Be able to design the software to use the gas sensor and its HMI & 3 & 3 \\ \hline
    Be able to combine all of the above mentioned components into a smart device & 4 & 4 \\ \hline
    \end{tabular}
    \caption{Skill matrix for Microcontrollers and Open Source Hardware}
    \label{table:skills-microcontrollers}
\end{table}

\begin{table}[H]
    \centering
    \arrayrulecolor{black} % Defines the color of the table borders
    \renewcommand{\arraystretch}{1.5} % Adjusts the vertical spacing between rows
    \begin{tabular}{|p{3.5cm}|p{8cm}|p{3.5cm}|}
    \hline
    \rowcolor[gray]{0.8}
    \textbf{Skill} & \textbf{Description} & \textbf{Level of Mastery} \\
    \hline
    Understand Microcontroller Architecture & 
    Understanding microcontroller architectures, functionalities, and use cases. & 
    Advanced \\
    \hline
    Data Acquisition System Design & 
    Designing end-to-end data acquisition systems, including sensor selection, signal conditioning, and microcontroller integration. & 
    Advanced \\
    \hline
    Sensor Signal Conditioner Circuit & 
    Developing and simulating signal conditioning circuits to ensure accurate data acquisition. & 
    Advanced \\
    \hline
    Shield Design for Gas Sensor & 
    Designing and fabricating custom hardware expansions (shields) for proper integration of gas sensors. & 
    Advanced \\
    \hline
    Sensor Software and HMI Development & 
    Implementing embedded software solutions, including sensor drivers, data processing, and user interfaces for gas sensing applications. & 
    Advanced \\
    \hline
    Smart Device Integration & 
    Combining hardware and software components into a cohesive, fully operational IoT device. & 
    Advanced \\
    \hline
    \end{tabular}
    \caption{Skills Acquired During the Microcontrollers and Open-Source Hardware Module}
\end{table}

\subsection{Analysis and remarks}

\indent \indent This course was very interesting to me because it allowed me to discover new tools such as KiCad and NodeRed, and to use them in an end-to-end project. We went from creating a sensor all the way to a web interface for data visualization, including designing a PCB. It was very rewarding to see the project come to life and to manage our time effectively to complete everything on schedule.

I also enjoyed helping classmates from the IR track, who did not have much experience with microcontrollers and were not always very comfortable with them.

As part of this course, I also learned about licensing concepts whether open source or proprietary and all their implications. This is a notion I consider very important, and it was the first time I had addressed it in my studies.

Finally, I got to learn more about the functioning and limitations of INSA's ChirpStack gateway, which prompted me to use my own LoRa gateway registered on the LoRaWAN TTN network.


\newpage
% Communication protocols for LP-WPAN
\section{Communication protocols for LP-WPAN}

\newpage
% Embedded IA for IOT
\section{Embedded IA for IOT}

\newpage
% Master REOC
\section{Master REOC}

\subsection{Context}
The Master REOC program is a 6 months program done in parallel of the 5ISS, the Thursdays afternoons. It is done in collaboration between INSA Toulouse and ENSEEIHT.
The objective of this program is to train network and telecommunications engineers, system/network architects, system/network administrators, and security engineers.

\subsection{Technical summary}
%Le Master REOC est structuré de manière à nous fournir à la fois des connaissances théoriques sur des problématiques spécifiques aux réseaux et télécommunications, et des
%compétences pratiques pour mettre en application les solutions disponibles pour les résoudre.
%Le Master se compose de 10 matières différentes, chacune abordant un aspect spécifique des réseaux et télécommunications.\\
    % Liste des cours
    %- Réseaux Embarqués et QoS
    %- Virtualisation dans l’embarqué
    %- Virtualisation de réseau / SDN
    %- Évaluation de performance Sans fil
    %- Virtualisation de réseau / NFV
    %- Virtualisation de réseau / Projet
    %- Architectures Modulaires
    %- Simulation avancée
    %- Pire cas avion
    %- Modélisation d’un Middleware IoT
    %- Modélisation d’un réseau LoRa"
%Nous avons abordés des sujets tels que la virtualisation, la qualité de service, la modélisation de réseaux, la simulation de réseaux, la sécurité des réseaux, la performance
%des réseaux sans fil, la modélisation de middleware IoT, la modélisation de réseaux LoRa,, mais aussi des sujets plus spécifiques comme les modules intégré de l'avionique ou l'analyse du temps de traversée pire cas sur les réseaux ethernet commutés.

The Master REOC is structured to provide both theoretical knowledge on specific networking and telecommunications challenges, as well as practical skills to implement available solutions to address these issues.
The program consists of 10 distinct courses, each focusing on a specific aspect of networking and telecommunications.
\\
We explored topics such as virtualization, quality of service (QoS), network modeling, network simulation, network security, wireless network performance, IoT middleware modeling, and LoRa network modeling.
Additionally, the program covered more specialized subjects, such as integrated avionics modules and worst-case traversal time analysis in switched Ethernet networks.

\subsection{Skills acquired}
Through this course, I consider to have acquired the following skills:

%- Conception de nouvelles solutions d'architectures réseau et de services associant l'ensemble des briques nécessaires (infrastructure, SI, réseau,....) en réponse à l'expression des besoins des opérateurs, entreprises, institutions privées ou publiques....
%- Réalisation d'une étude d'ingénierie détaillée afin de faire correspondre les déploiements locaux aux exigences de capacité, de couverture et de qualité de service définies dans le dossier d'architecture de communication.
%- Pilotage de l'implémentation des éléments de réseau et de l'intégration technique des équipements par les équipes opérationnelles suivant la nature des projets de déploiement.
%- Mise en service, paramétrage et configuration des équipements de réseaux, télécoms et services dans le cadre des installations prévues.
%- Respect du plan de prévention des risques et de l'application des règles de sécurité.
%- Supervision des systèmes de télécommunications, des équipements du réseau et des services au moyen des outils de supervision de son domaine.
%- Proposition, identification et définition des actions d'évolution et d'amélioration de service (à destination des équipes exploitation et/ou ingénierie).
%- Veille technologique et force de proposition sur de nouvelles fonctionnalités à ajouter aux solutions de services en développement.
\begin{table}[H]
    \centering
    \begin{tabular}{|p{3.5cm}|p{8cm}|p{3.5cm}|}
    \hline
    \textbf{Competence} & \textbf{Description} & \textbf{Level of Mastery} \\
    \hline
    \end{tabular}
    \caption{Competences Gained in Jean-Luc Scharbard courses}
\end{table}

\begin{table}[h!]
    \centering
    \begin{tabular}{|p{3.5cm}|p{8cm}|p{3.5cm}|}
    \hline
    \textbf{Competence} & \textbf{Description} & \textbf{Level of Mastery} \\
    \hline
    \end{tabular}
    \caption{Competences Gained in Jean-Luc Scharbard courses}
\end{table} 

\begin{table}[h!]
    \centering
    \begin{tabular}{|p{3.5cm}|p{8cm}|p{3.5cm}|}
    \hline
    \textbf{Competence} & \textbf{Description} & \textbf{Level of Mastery} \\
    \hline
    Defining and Deploying Virtual Networks & Designing and implementing virtual networks, ensuring proper communication between components. & Advanced \\
    \hline
    Working with SDN and NFV & Gained hands-on experience creating VNFs, managing SDN controllers, and addressing network challenges. & Advanced \\
    \hline
    Network Monitoring and Optimization & Implemented systems to track bandwidth and mitigate congestion through SDN rule adjustments. & Advanced\\
    \hline
    Collaborative Problem-Solving & Successfully worked in a team to design and implement a complex network simulation. & Advanced\\
    \hline
    \end{tabular}
    \caption{Competences Gained During the Master REOC}
\end{table}

\subsection{Analysis and remarks}
The Master REOC program was one of the most demanding experiences of my academic journey.
The practical sessions, particularly the Mininet project, required intense focus and teamwork to complete within the constraints of the lab.
While the workload was challenging, it was incredibly rewarding to see our theoretical knowledge applied to real-world scenarios.

This project helped me develop a deeper understanding of constrained networks and cutting-edge technologies like SDN and NFV.
The process of designing and simulating a network topology taught me the importance of planning and precision, while implementing VNFs and developing the Flask interface allowed me to enhance my technical and problem-solving skills.

If I had the opportunity to redo this year, I would organise myself better.
The challenges and time constraints, in parallel of my courses at INSA, were quite imposing, and pushed me to improve both my technical abilities and my ability to work under pressure.
In fact, I think that this was a great experience, but I would have liked to have more time to explore the different courses.

\newpage
\chapter{Conclusion}

% Dans ce portfolio, j'ai détaillé mon parcours académique et professionnel en mettant en avant les compétences techniques et transversales que j'ai acquises. Mon cursus à l'INSA, axé sur les systèmes intelligents, m'a permis d'explorer des domaines variés tel que l'Internet des Objets (IoT), la cybersécurité, l'IA pour les systemes embarqués, l'optimisation énergétique et les réseaux de télécommunication.

% Les nombreux projets réalisés cette année, comme la conception de capteur nanotechnologique à l'AIME, le développement de systèmes embarqués ou encore la mise en oeuvre de protocoles de communication, ont renforcé .
% Mon expérience internantionnale en Roumanie, et mes collaborations avec les équipes sur place et dans le monde, m'ont permis de développer des compétences en travail d'équipe, gestion de projets et en communication interculturelle.

% Fort de ces expériences, je souhaite poursuivre mes études en intégrant le Mastère Spécialisé TLS-SEC (Toulouse Security). Ce programme me permettra d'approfondir les connaissance acquises ces dernières années en sécurité, et de me spécialiser dans ce domaine en pleine expansion.

% Ce portfolio illustre non seulement mon parcours dans le supérieur, mais aussi mes compétences techniques et transversales, et mon projet professionnel.

\indent \indent In this portfolio, I have detailed my academic and professional background, highlighting the technical and transversal skills I have acquired. My studies at INSA, focused on intelligent systems, have allowed me to explore various fields such as the Internet of Things (IoT), cybersecurity, artificial intelligence for embedded systems, energy optimization, and telecommunication networks.
\\

The numerous projects I have completed this year, such as designing a nanotechnology sensor at AIME, developing embedded systems, and implementing communication protocols, have strengthened my technical skills and my ability to work in a team.
\\

My international experience in Romania, as well as my collaborations with teams both locally and worldwide, have enabled me to develop skills in teamwork, project management, and intercultural communication.
\\
Building on these experiences, I aim to further my studies by enrolling in the Mastère Spécialisé TLS-SEC (Toulouse Security). This program will allow me to deepen the knowledge I have acquired in security over the past few years and specialize in this rapidly expanding field.
\\

This portfolio not only illustrates my academic journey but also showcases my technical and transversal skills, as well as my professional aspirations.
\end{document}
