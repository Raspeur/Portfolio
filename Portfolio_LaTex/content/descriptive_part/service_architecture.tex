\section{Service Oriented Architecture - Software Engineering}
\label{sec:service_architecture}
\subsection{Context}
%Service Oriented architecture is a course taught by Nawal Guermouche in a MOOC format (Massive Open Online Course). The objective of this course is to discover legacy and actual architectures for software engineering.
%3 courses : \\
%The SOAP standard: In this course, you have studied the SOA architecture, the web service technology and its standards. \\
%RESTFull services: \\
%Microservices: How can monolithic applications can be limiting in developping flexible and scalable applications. I have studied SOA and REST architecture and what they bring to overcome the monolithic issues. Finally, I have discovered the Microservices architectures that extends the SOA architecture.
%As a student coming from an Automatic, Electronic and Informatic background it was very interesting to discovers the different standards architecture for software engineering.

\indent \indent The Service-Oriented Architecture (SOA) course was taught by Ms. Nawal Guermouche in the form of a MOOC.
It introduces software engineering architectures and traces the evolution of software systems, from monolithic applications to distributed and modular architectures such as SOA, RESTful services and microservices.
The course highlights the importance of this type of architecture in the creation of today's scalable, reusable and flexible software systems.
\vspace{0.25cm}

\noindent Thanks to this course, I discovered how the SOAP and WSDL protocols serve as the basis for SOA, enabling interoperability and modularity.
I also learned about RESTful services, which offer a simpler, lighter approach to web services by relying on HTTP methods for communication. 
And finally, we discovered the principles of microservices, where the decomposition of large applications into smaller, independent services improves scalability, fault isolation and deployment flexibility.
\vspace{0.25cm}

\noindent As a student with a background in automation and electronics, this course was a great discovery for me and gave me a new perspective on software system design. 

\subsection{Technical Summary} 
\indent \indent The course looked at the principles and implementation of three key software architectures: SOA, REST and microservices.
Service-oriented architecture (SOA) was presented as a modular design paradigm that encapsulates functionality in independent services, enabling loose coupling and reuse.
Key technologies, such as SOAP (Simple Object Access Protocol) and WSDL (Web Services Description Language), were explored to understand how services are defined, discovered and invoked, thus offering seamless interoperability between platforms.
\vspace{0.25cm}

\noindent REST (Representational State Transfer) was presented as a lightweight alternative to SOAP, emphasizing stateless communication and the use of standard HTTP methods like GET, POST, and DELETE for interacting with resources. 
The simplicity and scalability of REST make it a popular choice for modern web services. 
In contrast, SOAP's complexity is offset by its robustness and suitability for enterprise-level systems.
\vspace{0.25cm}

\noindent Building on these foundations, the course introduced microservices as an extension of SOA, focusing on decomposing monolithic applications into smaller, independent services aligned with business functionalities. 
Each microservice operates autonomously, often managing its own database and exposing APIs, typically using REST. This approach facilitates scalability, continuous deployment, and fault isolation, making it ideal for large-scale, dynamic systems.
\vspace{0.25cm}

\noindent To complement these architectural concepts, the course provided practical insights into tools and technologies such as Spring Boot for building microservices, Postman for API testing, and Microsoft Azure for deploying distributed systems. 
These tools demonstrated how theoretical principles are applied in real-world scenarios to design and manage modern software systems effectively.

\subsection{Practical work}
\indent \indent The practical component of the course was structured into two stages: guided tutorials and independent projects. This approach allowed us to progressively build on theoretical concepts, starting with foundational exercises and advancing to complex implementations.

\subsubsection{Guided Tutorials}
The initial phase involved a series of tutorials, each focusing on a specific aspect of SOA, RESTful services, and microservices:
\begin{itemize}
    \item SOAP: Introduction to SOAP web services, WSDL, and SOAP clients.
    \item REST: Setting up RESTful services, creating REST clients and handling data formats (e.g., JSON, XML)
    \item Microservices: Using Spring Boot and Spring Cloud to create services, manage service discovery, load balancing, configuration management, and client integration with configuration services.
\end{itemize}
These tutorials provided hands-on experience with tools and frameworks, helping us solidify our understanding of architectural principles.

\subsubsection{Projects: Volunteering Application and Smart Building Simulation}
After completing the tutorials, we applied our skills to two independent projects that simulated real-world scenarios:

\vspace{0.25cm}
\noindent The goal of the first project was to design RESTful microservices for a system where users could post requests for volunteers, volunteers could respond, and users could leave feedback.
Using Java, we implemented multiple services, each responsible for specific tasks like request management, volunteer interactions, and feedback processing. This project emphasized the importance of well-structured service communication and the use of REST APIs for seamless integration.
Smart Building Simulation:

\vspace{0.25cm}
\noindent In the second project, we created a distributed application that interfaced with a simulated sensor network for a smart building. This scenario included various rooms equipped with sensors (e.g., presence detectors) and actuators (e.g., doors, windows, alarms, lights).
We developed the following microservices using Java:
\begin{itemize}
    \item A sensor service to manage sensor data and events.
    \item An actuator service to control actuators based on sensor inputs.
    \item A room configuration service to handle building setups.
    \item A user service to determine the location of individuals within the building.
    \item A time service to synchronize operations across the system.
\end{itemize}

\noindent Additionally, we built a front-end interface using HTML, CSS, and JavaScript to visualize and interact with the smart building. A Python script simulated sensor behavior according to a predefined scenario, enhancing the realism of the simulation.
\vspace{0.25cm}

\noindent These projects provided invaluable experience in designing, implementing, and deploying distributed systems. The integration of a simulated sensor network and a custom front-end interface highlighted the complexity of real-world applications and allowed us to apply a full-stack development approach.

\subsection{Skills gained}
\indent \indent This course equipped me with both theoretical knowledge and practical skills necessary to design, deploy, and manage Service-Oriented Architectures (SOA). Specifically, I gained the following competences:

\begin{table}[h!]
    \centering
    \arrayrulecolor{black} % Defines the color of the table borders
    \renewcommand{\arraystretch}{1.5} % Adjusts the vertical spacing between rows
    \begin{tabular}{|p{11cm}|c|c|}
    \hline
    \rowcolor[gray]{0.8}
    \textbf{Skills} & \textbf{Required} & \textbf{Achieved} \\ \hline
    \rowcolor[gray]{0.9} \textbf{Software Engineering} &  &  \\ \hline
    Define the different phases in software development & 3 & 3 \\ \hline
    Know the different project management methods & 3 & 3 \\ \hline
    Apply one of these methods to a project & 3 & 3 \\ \hline
    \end{tabular}
    \caption{Skill matrix for Software Engineering}
    \label{table:skills-software-engineering}
\end{table}

\begin{table}[h!]
    \centering
    \arrayrulecolor{black} % Defines the color of the table borders
    \renewcommand{\arraystretch}{1.5} % Adjusts the vertical spacing between rows
    \begin{tabular}{|p{11cm}|c|c|}
    \hline
    \rowcolor[gray]{0.8}
    \textbf{Skills} & \textbf{Required} & \textbf{Achieved} \\ \hline
    \rowcolor[gray]{0.9} \textbf{Service Oriented Architecture} &  &  \\ \hline
    Know how to define a Service Oriented Architecture & 4 & 4 \\ \hline
    Deploy an SOA with web services & 4 & 4 \\ \hline
    Deploy and configure an SOA using SOAP & 4 & 4 \\ \hline
    Deploy and configure an SOA using REST & 4 & 4 \\ \hline
    Integrate a process manager in an SOA & 4 & 4 \\ \hline
    \end{tabular}
    \caption{Skill matrix for Service Oriented Architecture}
    \label{table:skills-soa}
\end{table}

\begin{table}[H]
    \centering
    \arrayrulecolor{black} % Defines the color of the table borders
    \renewcommand{\arraystretch}{1.5} % Adjusts the vertical spacing between rows
    \begin{tabular}{|p{3.5cm}|p{8cm}|c|}
    \hline
    \rowcolor[gray]{0.8}
    \textbf{Competence} & \textbf{Description} & \textbf{Level of Mastery} \\ \hline
    Defining a Service-Oriented Architecture & Ability to identify and design modular, loosely coupled services that promote reusability and scalability. & Advanced \\ \hline
    Deploying an SOA with Web Services       & Practical experience implementing SOA with web services to enable interoperability across platforms. & Advanced \\ \hline
    Configuring SOA with SOAP                & Skills in deploying SOAP-based services, including WSDL design, handling SOAP requests, and integration. & Intermediate \\ \hline
    Configuring SOA with REST                & Proficiency in creating RESTful services using HTTP methods and lightweight data formats like JSON and XML. & Advanced \\ \hline
    Integrating a Process Manager in SOA     & Knowledge of service orchestration and choreography to coordinate workflows across multiple services. & Intermediate \\ \hline
    \end{tabular}
    \caption{Competences Gained During the Service-Oriented Architecture Course}
\end{table}

\subsection{Analysis and remarks}

\indent \indent The Service-Oriented Architecture course was both challenging and rewarding. It gave me a solid introduction to SOAP, REST, and microservices, and helped me understand how these architectures are used in real-world software development. The combination of tutorials and projects made the learning process engaging, but it also required a lot of effort and time management.
\noindent The tutorials, while very informative, were particularly time-consuming. They required additional work at home to ensure I could complete the exercises and still have time for the projects.\\
Coming from the Automatic Electronics speciality, and even though it was demanding, I found the tutorials really necessary.  They covered key concepts such as service discovery and load balancing, which were crucial to the realization of the projects.
\vspace{0.25cm}

The two projects were where I really applied what I learned. In the first project, designing RESTful microservices for a volunteering application helped me grasp how to build APIs and manage service interactions. The second project, where we simulated a smart building, was even more complex. It was rewarding to see how sensors, actuators, and services could work together in a distributed system. It also showed me the challenges of coordinating multiple services and maintaining a functional system.
\noindent I found the comparison between SOAP and REST very useful. SOAP is highly reliable and standardized, but REST is much simpler and better suited for most web applications. Learning about microservices was exciting too, as it showed how small, independent services can make systems more flexible and scalable. At the same time, I realized the extra complexity they add, especially when managing multiple services.
\vspace{0.25cm}
\newpage
Overall, the course was intense but worthwhile. The combination of tutorials and projects gave me both theoretical knowledge and practical skills. It wasn't always easy to balance the workload, but the experience was definitely valuable and has prepared me to tackle real-world challenges in software development.