% \section{Apprenticeship at Vitesco Technologies}
% \subsection{Context}
% Au cours de ma 2ème année d'alternance chez Vitesco Technologies, mes projets se sont concentrés sur du développement logiciel embarqué, et plus préciseement de la conception de drivers generiques pour des microcontroleurs. J'ai par exemple cette année travaillé sur le développement d'un driver PWM pour un nouveau microcontrolleur de la famille Infineon AURIX, le TC4. J'ai également fais des tests d'intégration d'un projet que sur lequel j'avais travailé l'année passée, le Wake-Up Controller.
% Dans le cadre de la validation de mon diplome j'ai du realiser un stage à Iasi, en Roumanie, dans lequel j'ai travaillé à la première activation du harware security manager (HSM) d'un microcontrolleur de la famille Renesas.

% \subsection{Technical Summary}
% \indent \indent Le développement du pilote PWM pour le microcontroleur TC4 d'Infineon est la seule nouvelle tache que j'ai réalisé dans l'équipe Complexe Device Driver (CDD) cette année. J'ai du concevoir un driver générique qui permettrait de controler les signaux PWM de manière simple et efficace. J'ai également réalisé des tests unitaires et d'intégrations pour valider le bon fonctionnement du "Wake-Up Controller", un projet sur lequel j'avais travaillé l'année précédente. La fonction principale de ce module est de réveiller le microcontrolleur principal d'un etat de veille profonde, ou d'un arret complet. Pour ce faire, j'ai conçu et appliqués des scénarios de test pour valider les déclenchements de réveil à partir de signaux analogiques.

% Enfin, j'ai réalisé un stage de 3 mois à Iasi, en Roumanie, dans lequel j'ai travaillé sur l'activation du Hardware Security Manager (HSM) d'un microcontrolleur de la famille Renesas, le RH850. J'ai du comprendre le fonctionnement de ce module, et en effectué la première activation, en se garantissant la possibilité de le déverouiller. Car en théorie, une fois le HSM activé, il n'est plus possible de reprogrammer le microcontrolleur.

% \subsection{Skills Acquired}
% Au cours de mon apprentissage, j'ai développé différentes compétences techniques et professionnelles :
% \begin{table}[H]
%     \centering
%     \arrayrulecolor{black} % Defines the color of the table borders
%     \renewcommand{\arraystretch}{1.5} % Adjusts the vertical spacing between rows
%     \begin{tabular}{|p{3.5cm}|p{8cm}|c|}
%         \hline
%         \rowcolor[gray]{0.8}
%         \textbf{Skill} & \textbf{Description} & \textbf{Level of Mastery} \\
%         \hline
%         Embedded Systems Development & Proficient in designing and implementing drivers for embedded systems (e.g., PWM driver development). & Advanced \\
%         \hline
%         Microcontroller Programming & Advanced knowledge of microcontroller architectures (e.g., Infineon TC4, Renesas RH850). & Advanced \\
%         \hline
%         Software Validation & Expertise in unit, integration, and black-box testing of embedded modules. & Advanced \\
%         \hline
%         Hardware Security Implementation & Hands-on experience activating and configuring Hardware Security Modules (HSM) for secure communication. & Intermediate \\
%         \hline
%         Low-Level Programming & Proficient in writing and debugging low-level code, including linker configuration and interrupt handling. & Advanced \\
%         \hline
%         Collaboration and Project Management & Developed teamwork and coordination skills through international and cross-functional collaboration. & Advanced \\
%         \hline
%         Software Engineering Principles & Applied modular and scalable development practices, including BuildUnit management. & Advanced \\
%         \hline
%         Cross-Functional Communication & Experience conveying technical details to multidisciplinary teams (e.g., firmware, hardware, and testing groups). & Advanced \\
%         \hline
%         Documentation and Reporting & Created technical documentation and test reports for internal validation and project deliverables. & Advanced \\
%         \hline
%         International Work Experience & Gained cultural and professional exposure by working with multinational teams. & Intermediate \\
%         \hline
%     \end{tabular}
%     \caption{Skills Acquired During the Apprenticeship at Vitesco Technologies}
% \end{table}

% \subsection{Analysis and remarks}

% \indent \indent %Avoir fait le choix de l'alternance en 3ème année  m'a permis de découvrir la réalité du monde industriel, et de l'importance des cours qui sont enseignées à l'INSA. J'ai pu à la fois mettre en pratque des connaissances acquises à l'école, tout en gagnant de nouvelles compétences techniques et professionnelles.
% %J'ai pu travailler sur des projets variés, allant du développement de drivers pour microcontrolleurs, à l'activation d'un module de sécurité matérielle sur un microcontrolleur. J'ai également eu l'opportunité de travailler à l'étranger, en Roumanie, ce qui m'a permis de découvrir une nouvelle culture et de travailler avec des équipes internationales.

% Avoir choisi l'alternance en 3A a été une étape déterminante dans mon parcours. Cette expérience m'a permis de comprendre concrètement la réalité du monde industriel et d'apprécier davantage l'importance des cours dispensés à l'INSA. J'ai pu mettre en pratique les connaissances acquises à l'école tout en développant de nouvelles compétences, à la fois techniques et professionnelles. Les projets auxquels j'ai participé ont couvert un large éventail de domaines, allant du développement de drivers pour microcontrôleurs à l'activation d'un module de sécurité matérielle. De plus, mon séjour en Roumanie pour travailler avec les équipes de Iasi m'a offert une immersion culturelle et professionnelle enrichissante, et m'a permis de collaborer avec des spécialistes internationaux.

% Le projet d'activation du Hardware Security Manager (HSM) m'a particulièrement intéréssé, car il m'a confronté aux défis de la conception des systèmes sécurisés et m'a permis de voir comment sont appliqués les principes de sécurité étudiés en cours. J'ai également appris à mieux communiquer et à gérer la coordination de projets avec des équipes dispersées géographiquement, ce qui a renforcé mes compétences en gestion de projet et en collaboration internationale. La prise en main d'un nouveau microcontrôleur, comme le RH850 de Renesas, a représenté une courbe d'apprentissage exigeante, qui a nécessité persévérance et travail d'équipe pour être surmontée.

% Enfin, concilier les exigences académiques et professionnelles a renforcé ma capacité à gérer mon temps et à définir mes priorités. Vous comprenez donc pourquoi je considère mon alternance chez Vitesco Technologies comme un tournant majeur dans mon parcours d'ingénieur : elle m'a permis d'approfondir mon savoir-faire technique, d'élargir ma vision du travail en équipe à l'international et de confirmer mon intérêt pour la cyber et la sécurité.

\section{Apprenticeship at Vitesco Technologies}
\subsection{Context}
\indent \indent During my second year of apprenticeship at Vitesco Technologies, my projects focused on embedded software development, specifically on the design of generic drivers for microcontrollers. For instance, this year I worked on developing a PWM driver for a new microcontroller in Infineon's AURIX family, the TC4. I also carried out integration tests for a project I had worked on the previous year: the Wake-Up Controller.  
As part of my degree requirements, I had to complete an internship in Iasi, Romania, where I worked on the initial activation of the Hardware Security Manager (HSM) for a Renesas microcontroller.

\subsection{Technical Summary}
\indent \indent Developing the PWM driver for Infineon's TC4 microcontroller was the only new task I took on within the Complex Device Driver (CDD) team this year. I had to design a generic driver that would enable simple and efficient control of PWM signals. I also performed unit and integration tests to validate the proper functioning of the \textit{Wake-Up Controller}, a project I had contributed to the previous year. The main function of this module is to wake the main microcontroller from a deep sleep state or from a complete shutdown. To achieve this, I developed and applied test scenarios to validate wake-up triggers based on analog signals.
\vspace{0.25cm}

\noindent Finally, I completed a three-month internship in Iasi, Romania, during which I worked on activating the Hardware Security Manager (HSM) for a Renesas microcontroller, the RH850. I needed to understand how this module operates and carry out its initial activation, while ensuring it could still be unlocked. Indeed, in theory, once the HSM is activated, reprogramming the microcontroller is no longer possible.

\subsection{Skills Acquired}
Throughout my apprenticeship, I developed a variety of technical and professional skills:
\begin{table}[H]
    \centering
    \arrayrulecolor{black} % Defines the color of the table borders
    \renewcommand{\arraystretch}{1.5} % Adjusts the vertical spacing between rows
    \begin{tabular}{|p{3.5cm}|p{8cm}|c|}
        \hline
        \rowcolor[gray]{0.8}
        \textbf{Skill} & \textbf{Description} & \textbf{Level of Mastery} \\
        \hline
        Embedded Systems Development & Proficient in designing and implementing drivers for embedded systems (e.g., PWM driver development). & Advanced \\
        \hline
        Microcontroller Programming & Advanced knowledge of microcontroller architectures (e.g., Infineon TC4, Renesas RH850). & Advanced \\
        \hline
        Software Validation & Expertise in unit, integration, and black-box testing of embedded modules. & Advanced \\
        \hline
        Hardware Security Implementation & Hands-on experience activating and configuring Hardware Security Modules (HSM) for secure communication. & Intermediate \\
        \hline
        Low-Level Programming & Proficient in writing and debugging low-level code, including linker configuration and interrupt handling. & Advanced \\
        \hline
        Collaboration and Project Management & Developed teamwork and coordination skills through international and cross-functional collaboration. & Advanced \\
        \hline
        Software Engineering Principles & Applied modular and scalable development practices, including BuildUnit management. & Advanced \\
        \hline
        Cross-Functional Communication & Experience conveying technical details to multidisciplinary teams (e.g., firmware, hardware, and testing groups). & Advanced \\
        \hline
        Documentation and Reporting & Created technical documentation and test reports for internal validation and project deliverables. & Advanced \\
        \hline
        International Work Experience & Gained cultural and professional exposure by working with multinational teams. & Intermediate \\
        \hline
    \end{tabular}
    \caption{Skills acquired since the start of my apprenticeship at Vitesco Technologies}
\end{table}

\subsection{Analysis and remarks}
\indent \indent Choosing an apprenticeship in my third year was a pivotal step in my journey. This experience allowed me to gain firsthand insight into the realities of industry and to better appreciate the importance of the courses taught at INSA. I was able to apply the knowledge acquired at school while developing new technical and professional skills. The projects I worked on covered a wide range of areas, from microcontroller driver development to activating a hardware security module. Furthermore, my stay in Romania, working with the Iasi teams, provided me with a rich cultural and professional immersion and allowed me to collaborate with international experts.
\vspace{0.25cm}

\noindent I found the Hardware Security Manager (HSM) activation project particularly interesting because it challenged me with secure system design and showed how the security principles studied in class are put into practice. I also learned to communicate more effectively and coordinate with geographically dispersed teams, which strengthened my project management and international collaboration skills. Getting up to speed with a new microcontroller such as the Renesas RH850 was a steep learning curve that required perseverance and teamwork to overcome.
\vspace{0.25cm}

\noindent Finally, balancing academic and professional obligations enhanced my time management skills and helped me to set clear priorities. This is why I consider my apprenticeship at Vitesco Technologies a major turning point in my engineering career: it allowed me to deepen my technical expertise, broaden my perspective on international teamwork, and confirm my interest in cybersecurity and security.
