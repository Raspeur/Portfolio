% \subsection*{Presentation}
% Je suis Clément Gauché un étudiant en 5ème année à l'Institut National des Sciences Appliquées de Toulouse, en spécialité Innovative Smart System.\\
% Dans ce portfolio je présente les compétences et expériences que j'ai acquises dans cette spécialité. Il a pour but d'illuster la manère dont mes expériences professionnelles et académiques m'ont permis de développer une expertise technique et des compétences transversales.

% Ce protfolio met l'accent sur les cours suivis, les projets innovants réalisés et l'alternance effectuée en entreprise. Dans chaque matière, j'essai d'exprimer ce que cela m'a apporté et de quelle manière elle s'inscrit dans mon parcours académique et professionnel.


% \subsection*{Organisation du portfolio}
% Ce portfolio est organisé de la fmanière à fournir une vue d'ensemble claire et accessible de mes projets académiques et professionnels.\\
% Chaque section correspond a une matière, et est subdivisée en plusieurs parties clé:
% \begin{itemize}
%     \item Contexte : Introduction à la matière ou au projet, expliquant les objectifs et le cadre général.
%     \item Technical summary : Un résumé des concept théorique et technique abordés.
%     \item Pratical Work : Une description des activités pratiques et projets réalisés.
%     \item Skills Acquired : Une présentation des compétences développées, illustrées par des tableaux dédiés.
%     \item Analysis and remarks : 
% \end{itemize}


% Les compétences acquises sont classées dans 2 types de tableaux. Le premier type, contenant les colonnes "Required" et "Achieved", est un tableau de compétences en auto-évaluation fourni par les professeurs.

% \begin{table}[h!]
%     \centering
%     \arrayrulecolor{black} % Définit la couleur des bordures du tableau
%     \renewcommand{\arraystretch}{1.5} % Ajuste l'espacement vertical des lignes
%     \begin{tabular}{|p{11cm}|c|c|}
%     \hline
%     \rowcolor[gray]{0.8}
%     \textbf{Skills} & \textbf{Required} & \textbf{Achieved} \\ \hline
%     \rowcolor[gray]{0.9} \textbf{Self evaluation with portfolio} &  &  \\ \hline
%     Reflect upon my training process and methods & 3 & 4 \\ \hline
%     Be able to put forward my training experiences, whether they be explicit or implicit & 3 & 4 \\ \hline
%     Be self-sufficient and responsible towards my education & 4 & 4 \\ \hline
%     \end{tabular}
%     \caption{Skill matrix for the portfolio management and self-evaluation}
%     \label{table:skills}
% \end{table}

% Le second type, contenant la colonne "Level of Mastery", est un tableau de compétences d'auto-évaluation complémentaires que je considère avoir acquis.

\subsection*{Presentation}
My name is Clément Gauché, and I am a fifth-year student at the Institut National des Sciences Appliquées de Toulouse, specializing in Innovative Smart Systems.\\
In this portfolio, I present the skills and experiences I have acquired in this field. Its purpose is to illustrate how my professional and academic experiences have enabled me to develop both technical expertise and cross-disciplinary skills.

This portfolio highlights the courses I have taken, the innovative projects I have completed, and the work-study program I undertook in a company. For each subject, I aim to express what I have gained from it and how it fits into my academic and professional journey.

\subsection*{Portfolio Organization}
This portfolio is organized to provide a clear and accessible overview of my academic and professional projects.\\
Each section corresponds to a specific subject and is divided into several key parts:
\begin{itemize}
    \item Context: Introduction to the subject or project, explaining the objectives and general framework.
    \item Technical Summary: A summary of the theoretical and technical concepts covered.
    \item Practical Work: A description of the practical activities and projects undertaken.
    \item Skills Acquired: A presentation of the skills developed, illustrated with dedicated tables.
    \item Analysis and Remarks: Reflections and feedback on the course.
\end{itemize}

The skills acquired are categorized into two types of tables. The first type, containing the columns "Required" and "Achieved," is a self-assessment skills table provided by the instructors.

\begin{table}[h!]
    \centering
    \arrayrulecolor{black} % Sets the border color of the table
    \renewcommand{\arraystretch}{1.5} % Adjusts the vertical spacing of rows
    \begin{tabular}{|p{11cm}|c|c|}
    \hline
    \rowcolor[gray]{0.8}
    \textbf{Skills} & \textbf{Required} & \textbf{Achieved} \\ \hline
    \rowcolor[gray]{0.9} \textbf{Self-evaluation with portfolio} &  &  \\ \hline
    Reflect upon my training process and methods & 3 & 4 \\ \hline
    Be able to put forward my training experiences, whether they are explicit or implicit & 3 & 4 \\ \hline
    Be self-sufficient and responsible towards my education & 4 & 4 \\ \hline
    \end{tabular}
    \caption{Skill matrix for portfolio management and self-evaluation}
    \label{table:skills}
\end{table}

The second type, containing the column "Level of Mastery," is a complementary self-assessment skills table that reflects the skills I consider to have acquired.
