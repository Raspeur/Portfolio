\section{Master REOC}

\subsection{Context}
The Master REOC program is a 6 months program done in parallel of the 5ISS, the Thursdays afternoons. It is done in collaboration between INSA Toulouse and ENSEEIHT.
The objective of this program is to train network and telecommunications engineers, system/network architects, system/network administrators, and security engineers.

\subsection{Technical summary}
%Le Master REOC est structuré de manière à nous fournir à la fois des connaissances théoriques sur des problématiques spécifiques aux réseaux et télécommunications, et des
%compétences pratiques pour mettre en application les solutions disponibles pour les résoudre.
%Le Master se compose de 10 matières différentes, chacune abordant un aspect spécifique des réseaux et télécommunications.\\
    % Liste des cours
    %- Réseaux Embarqués et QoS
    %- Virtualisation dans l’embarqué
    %- Virtualisation de réseau / SDN
    %- Évaluation de performance Sans fil
    %- Virtualisation de réseau / NFV
    %- Virtualisation de réseau / Projet
    %- Architectures Modulaires
    %- Simulation avancée
    %- Pire cas avion
    %- Modélisation d’un Middleware IoT
    %- Modélisation d’un réseau LoRa"
%Nous avons abordés des sujets tels que la virtualisation, la qualité de service, la modélisation de réseaux, la simulation de réseaux, la sécurité des réseaux, la performance
%des réseaux sans fil, la modélisation de middleware IoT, la modélisation de réseaux LoRa,, mais aussi des sujets plus spécifiques comme les modules intégré de l'avionique ou l'analyse du temps de traversée pire cas sur les réseaux ethernet commutés.

The Master REOC is structured to provide both theoretical knowledge on specific networking and telecommunications challenges, as well as practical skills to implement available solutions to address these issues.
The program consists of 10 distinct courses, each focusing on a specific aspect of networking and telecommunications.
\\
We explored topics such as virtualization, quality of service (QoS), network modeling, network simulation, network security, wireless network performance, IoT middleware modeling, and LoRa network modeling.
Additionally, the program covered more specialized subjects, such as integrated avionics modules and worst-case traversal time analysis in switched Ethernet networks.

\subsection{Skills acquired}
Through this course, I consider to have acquired the following skills:

%- Conception de nouvelles solutions d'architectures réseau et de services associant l'ensemble des briques nécessaires (infrastructure, SI, réseau,....) en réponse à l'expression des besoins des opérateurs, entreprises, institutions privées ou publiques....
%- Réalisation d'une étude d'ingénierie détaillée afin de faire correspondre les déploiements locaux aux exigences de capacité, de couverture et de qualité de service définies dans le dossier d'architecture de communication.
%- Pilotage de l'implémentation des éléments de réseau et de l'intégration technique des équipements par les équipes opérationnelles suivant la nature des projets de déploiement.
%- Mise en service, paramétrage et configuration des équipements de réseaux, télécoms et services dans le cadre des installations prévues.
%- Respect du plan de prévention des risques et de l'application des règles de sécurité.
%- Supervision des systèmes de télécommunications, des équipements du réseau et des services au moyen des outils de supervision de son domaine.
%- Proposition, identification et définition des actions d'évolution et d'amélioration de service (à destination des équipes exploitation et/ou ingénierie).
%- Veille technologique et force de proposition sur de nouvelles fonctionnalités à ajouter aux solutions de services en développement.
\begin{table}[H]
    \centering
    \begin{tabular}{|p{3.5cm}|p{8cm}|p{3.5cm}|}
    \hline
    \textbf{Competence} & \textbf{Description} & \textbf{Level of Mastery} \\
    \hline
    \end{tabular}
    \caption{Competences Gained in Jean-Luc Scharbard courses}
\end{table}

\begin{table}[h!]
    \centering
    \begin{tabular}{|p{3.5cm}|p{8cm}|p{3.5cm}|}
    \hline
    \textbf{Competence} & \textbf{Description} & \textbf{Level of Mastery} \\
    \hline
    \end{tabular}
    \caption{Competences Gained in Jean-Luc Scharbard courses}
\end{table} 

\begin{table}[h!]
    \centering
    \begin{tabular}{|p{3.5cm}|p{8cm}|p{3.5cm}|}
    \hline
    \textbf{Competence} & \textbf{Description} & \textbf{Level of Mastery} \\
    \hline
    Defining and Deploying Virtual Networks & Designing and implementing virtual networks, ensuring proper communication between components. & Advanced \\
    \hline
    Working with SDN and NFV & Gained hands-on experience creating VNFs, managing SDN controllers, and addressing network challenges. & Advanced \\
    \hline
    Network Monitoring and Optimization & Implemented systems to track bandwidth and mitigate congestion through SDN rule adjustments. & Advanced\\
    \hline
    Collaborative Problem-Solving & Successfully worked in a team to design and implement a complex network simulation. & Advanced\\
    \hline
    \end{tabular}
    \caption{Competences Gained During the Master REOC}
\end{table}

\subsection{Analysis and remarks}
The Master REOC program was one of the most demanding experiences of my academic journey.
The practical sessions, particularly the Mininet project, required intense focus and teamwork to complete within the constraints of the lab.
While the workload was challenging, it was incredibly rewarding to see our theoretical knowledge applied to real-world scenarios.

This project helped me develop a deeper understanding of constrained networks and cutting-edge technologies like SDN and NFV.
The process of designing and simulating a network topology taught me the importance of planning and precision, while implementing VNFs and developing the Flask interface allowed me to enhance my technical and problem-solving skills.

If I had the opportunity to redo this year, I would organise myself better.
The challenges and time constraints, in parallel of my courses at INSA, were quite imposing, and pushed me to improve both my technical abilities and my ability to work under pressure.
In fact, I think that this was a great experience, but I would have liked to have more time to explore the different courses.