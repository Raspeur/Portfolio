\section{5G technologies}
\subsection{Context}

\indent \indent The 5G technologies module presented by Professor Etienne Sicard, focused on the rapid evolution of cellular networks, emphasizing the transformative nature of 5G. A unique aspect of the learning approach was the use of reverse pedagogy, where students presented the majority of the content. The central goal was to explore cutting-edge topics in mobile communications, ranging from modulation techniques to the societal impacts of 5G and beyond.
\vspace{0.25cm}

\noindent 5G represents a paradigm shift in mobile networks, introducing Software Defined Radio (SDR) and microservices, which revolutionized network architecture by increasing flexibility and efficiency. Through this course, we students, were encouraged to examine the current state of cellular technology while considering its implications for the future.

\subsection{Technical and practical work}

\indent \indent During this course we could discuss about many varied subject. I collaborated with Noel Jumin on a presentation titled "Samsung's Vision for 6G," analyzing how Samsung's ambitions for 6G diverged from its competitors. We highlighted Samsung's focus on sub-THz frequencies, AI integration, and next-generation services such as holographic communications. This required an in-depth understanding of how 5G technologies paved the way for these future advancements, including the challenges posed by scaling up.
\vspace{0.25cm}

\noindent Other students' presentations covered topics such as 5G modulation techniques, vehicular networks, and the environmental and societal implications of 5G and 6G. These presentations provided a comprehensive understanding of both the technical and non-technical aspects of mobile communications.

\subsection{Skills acquired}
\begin{table}[h!]
    \centering
    \arrayrulecolor{black} % Defines the color of the table borders
    \renewcommand{\arraystretch}{1.5} % Adjusts the vertical spacing between rows
    \begin{tabular}{|p{3.5cm}|p{8cm}|c|}
    \hline
    \rowcolor[gray]{0.8}
    \textbf{Competence} & \textbf{Description} & \textbf{Level of Mastery} \\
    \hline
    Mobile Communications Development & Understanding the major development phases for mobile communications and development of the associated technology. & Advanced \\
    \hline
    Impact of New Mobile Technology & Understanding the impact of new mobile technology. & Intermediate \\
    \hline
    \end{tabular}
    \caption{Competences Gained During the 5G Technologies Module}
\end{table}

\subsection{Analysis and remarks}
\indent \indent This course was highly engaging as it dealt with contemporary and practical topics in cellular network technology. The discussion of 5G as a foundational step towards 6G was particularly fascinating, showing how each generation builds upon the previous one. The reverse pedagogy approach fostered a dynamic and interactive learning environment.
\vspace{0.25cm}

\noindent However, I found reverse pedagogy less effective overall. While preparing and delivering our own presentations allowed us to master our chosen topic, it made it harder to fully comprehend the topics presented by others. During the presentations, my focus was often divided between understanding the other groups' content and refining my own work.
\vspace{0.25cm}

\noindent I believe that a better balance between lectures by the teacher and student presentations would enhance the learning experience. Teacher-led lectures could provide a deeper dive into each topic, ensuring a more comprehensive understanding, while our presentations could supplement this with detailed case studies or specific insights. As it stands, I feel I gained high-level knowledge of the subjects presented by others, but lacked the depth that direct teaching could have provided.