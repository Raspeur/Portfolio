\section{Service Architecture - Software Engineering}
\subsection{Context}
%Service Oriented architecture is a course taught by Nawal Guermouche in a MOOC format (Massive Open Online Course). The objective of this course is to discover legacy and actual architectures for software engineering.
%3 courses : \\
%The SOAP standard: In this course, you have studied the SOA architecture, the web service technology and its standards. \\
%RESTFull services: \\
%Microservices: How can monolithic applications can be limiting in developping flexible and scalable applications. I have studied SOA and REST architecture and what they bring to overcome the monolithic issues. Finally, I have discovered the Microservices architectures that extends the SOA architecture.
%As a student coming from an Automatic, Electronic and Informatic background it was very interesting to discovers the different standards architecture for software engineering.

\indent \indent The Service-Oriented Architecture (SOA) course, taught by Ms. Nawal Guermouche in a MOOC format, introduced software engineering architectures.
It traced the evolution of software systems from monolithic applications to distributed and modular architectures such as SOA, RESTful services and microservices.
The course highlighted the importance of these architectures in the creation of today's scalable, reusable and flexible software systems.

\indent Through this course, I discovered how SOAP standards serve as a foundation for SOA, enabling interoperability and modularity through the use of protocols such as WSDL and SOAP. 
I also learned about RESTful services, which offer a simpler, lighter approach to web services by relying on HTTP methods for communication. 
The course concluded with a study of microservices, where the focus was on decomposing large applications into smaller, independent services to improve scalability, fault isolation and deployment flexibility.

\indent As a student with a background in Automation, Electronics, and Informatics, this course provided a new perspective on software system design. 

\subsection{Technical}
\indent \indent The course looked at the principles and implementation of three key software architectures: SOA, REST and microservices.
Service-oriented architecture (SOA) was presented as a modular design paradigm that encapsulates functionality in independent services, enabling loose coupling and reuse.
Key technologies, such as SOAP (Simple Object Access Protocol) and WSDL (Web Services Description Language), were explored to understand how services are defined, discovered and invoked, thus offering seamless interoperability between platforms.

REST (Representational State Transfer) was presented as a lightweight alternative to SOAP, emphasizing stateless communication and the use of standard HTTP methods like GET, POST, and DELETE for interacting with resources. 
The simplicity and scalability of REST make it a popular choice for modern web services. 
In contrast, SOAP’s complexity is offset by its robustness and suitability for enterprise-level systems.

Building on these foundations, the course introduced microservices as an extension of SOA, focusing on decomposing monolithic applications into smaller, independent services aligned with business functionalities. 
Each microservice operates autonomously, often managing its own database and exposing APIs, typically using REST. This approach facilitates scalability, continuous deployment, and fault isolation, making it ideal for large-scale, dynamic systems.

To complement these architectural concepts, the course provided practical insights into tools and technologies such as Spring Boot for building microservices, Postman for API testing, and Microsoft Azure for deploying distributed systems. 
These tools demonstrated how theoretical principles are applied in real-world scenarios to design and manage modern software systems effectively.

\subsection{Practical work = Projets de TP}
The practical component of the course was structured into two stages: guided tutorials and independent projects. This approach allowed us to progressively build on theoretical concepts, starting with foundational exercises and advancing to complex implementations.
\\
\subsubsection{Guided Tutorials}
The initial phase involved a series of tutorials, each focusing on a specific aspect of SOA, RESTful services, and microservices:
\begin{itemize}
    \item SOAP: Introduction to SOAP web services, WSDL, and SOAP clients.
    \item REST: Setting up RESTful services, creating REST clients, handling data formats (e.g., JSON, XML), and exploring HATEOAS.
    \item Microservices: Using Spring Boot and Spring Cloud to create services, manage service discovery, load balancing, configuration management, and client integration with configuration services.
\end{itemize}
These tutorials provided hands-on experience with tools and frameworks, helping us solidify our understanding of architectural principles.

\subsubsection{Projects: Volunteering Application and Smart Building Simulation}
After completing the tutorials, we applied our skills to two independent projects that simulated real-world scenarios:

\vspace{0.25cm}
The goal of the first project was to design RESTful microservices for a system where users could post requests for volunteers, volunteers could respond, and users could leave feedback.
Using Java, we implemented multiple services, each responsible for specific tasks like request management, volunteer interactions, and feedback processing. This project emphasized the importance of well-structured service communication and the use of REST APIs for seamless integration.
Smart Building Simulation:

\vspace{0.25cm}
In the second project, we created a distributed application that interfaced with a simulated sensor network for a smart building. This scenario included various rooms equipped with sensors (e.g., presence detectors) and actuators (e.g., doors, windows, alarms, lights).
We developed the following microservices using Java:
\begin{itemize}
    \item A sensor service to manage sensor data and events.
    \item An actuator service to control actuators based on sensor inputs.
    \item A room configuration service to handle building setups.
    \item A user service to determine the location of individuals within the building.
    \item A time service to synchronize operations across the system.
\end{itemize}

Additionally, we built a front-end interface using HTML, CSS, and JavaScript to visualize and interact with the smart building. A Python script simulated sensor behavior according to a predefined scenario, enhancing the realism of the simulation.
\\
These projects provided invaluable experience in designing, implementing, and deploying distributed systems. The integration of a simulated sensor network and a custom front-end interface highlighted the complexity of real-world applications and allowed us to apply a full-stack development approach.

\subsection{Skills gained}
This course equipped me with both theoretical knowledge and practical skills necessary to design, deploy, and manage Service-Oriented Architectures (SOA). Specifically, I gained the following competences:

\begin{table}[h!]
    \centering
    \begin{tabular}{|p{3.5cm}|p{8cm}|p{3.5cm}|}
    \hline
    \textbf{Competence} & \textbf{Description} & \textbf{Level of Mastery} \\ \hline
    Defining a Service-Oriented Architecture & Ability to identify and design modular, loosely coupled services that promote reusability and scalability. & Advanced \\ \hline
    Deploying an SOA with Web Services       & Practical experience implementing SOA with web services to enable interoperability across platforms. & Advanced \\ \hline
    Configuring SOA with SOAP                & Skills in deploying SOAP-based services, including WSDL design, handling SOAP requests, and integration. & Intermediate \\ \hline
    Configuring SOA with REST                & Proficiency in creating RESTful services using HTTP methods and lightweight data formats like JSON and XML. & Advanced \\ \hline
    Integrating a Process Manager in SOA     & Knowledge of service orchestration and choreography to coordinate workflows across multiple services. & Intermediate \\ \hline
    \end{tabular}
    \caption{Competences Gained During the Service-Oriented Architecture Course}
\end{table}

\subsection{Analysis and remarks}

The Service-Oriented Architecture course was both challenging and rewarding. It gave me a solid introduction to SOAP, REST, and microservices, and helped me understand how these architectures are used in real-world software development. The combination of tutorials and projects made the learning process engaging, but it also required a lot of effort and time management.

The tutorials, while very informative, were particularly time-consuming. They required additional work at home to ensure I could complete the exercises and still have time for the projects. Even though this was demanding, the tutorials were essential in building a strong technical foundation. They covered key concepts like service discovery, load balancing, and HATEOAS, which were crucial for understanding the projects.

The two projects were where I really applied what I learned. In the first project, designing RESTful microservices for a volunteering application helped me grasp how to build APIs and manage service interactions. The second project, where we simulated a smart building, was even more complex. It was rewarding to see how sensors, actuators, and services could work together in a distributed system. It also showed me the challenges of coordinating multiple services and maintaining a functional system.

I found the comparison between SOAP and REST very useful. SOAP is highly reliable and standardized, but REST is much simpler and better suited for most web applications. Learning about microservices was exciting too, as it showed how small, independent services can make systems more flexible and scalable. At the same time, I realized the extra complexity they add, especially when managing multiple services.

Overall, the course was intense but worthwhile. The combination of tutorials and projects gave me both theoretical knowledge and practical skills. It wasn’t always easy to balance the workload, but the experience was definitely valuable and has prepared me to tackle real-world challenges in software development.