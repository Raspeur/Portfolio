% \section{LAB at AIME}
% \subsection{Context}
% \indent \indent L'INSA Toulouse possède un Atelier Interdisciplinaire de Microélectronique et d'Électronique (AIME) qui est un laboratoire de recherche et de formation en microélectronique et électronique. Dans le cadre de la formation ISS, nous avons passé y avons passé une semaine pour réaliser un capteur de gaz par synthèse chimique de nanoparticules $WO_3$ et leur intégration sur des dispositifs de micro-électronique. Ce laboratoire est équipé de salles blanches permettant d'effectuer des manipulations de haute précision nécéssaire pour la fabrication à la micro-echelle.

% Cependant, les étapes initiales du processus de fabrication, jusqu'à la photolithographie numéro 2 (« Contacts Opening »), avaient déjà été réalisées par les techniciens du laboratoire pour nous permettre de nous concentrer sur les étapes suivantes.

% \subsection{Technical Summary}
% Le projet visait à produire un capteur de gaz basé sur des nanoparticules de $WO_3$, un matériau semi-conducteur connu pour ses propriétés sensibles aux gaz. Les étapes principales étaient dans l'ordre de : 
% \begin{enumerate}
%     \item Réaliser la photolithographie numéro 3 pour le gravage des électrodes métalliques en aluminium.
%     \item Assembler des dispositifs, incluant le découpage et le montage des puces sur des supports TO5.
%     \item Synthétiser des nanoparticules de $WO_3$, comprenant la préparation des graines et la croissance de nanofils par des procédés hydrothermaux.
%     \item Intégrer des nano bâtonnets de $WO_3$ sur les peignes interdigités via la diélectrophorèse.
%     \item Faire caractérisation électrique des capteurs dans une atmosphère contrôlée, contenant de l'éthanol ou de l'ammoniaque.
% \end{enumerate}

% \subsection{Practical work}
% Au cours de la semaine nous avons pu réaliser les étapes 1 à 5 du TP. Nous travaillions en groupe de 4, et chaque groupe était responsable de son capteur. J'ai été en charge de la synthèse des nanoparticules de $WO_3$ et de leur intégration sur les peignes interdigités. J'ai également participé à la caractérisation électrique des capteurs.

% \subsection{Skills acquired}
% \indent \indent Grâce à ce laboratoire, je considère pense avoir acquis les compétences suivantes :
% \begin{table}[H]
%     \centering
%     \begin{tabular}{|p{5cm}|p{8cm}|p{3cm}|}
%     \hline
%     \textbf{Compétence} & \textbf{Description} & \textbf{Niveau de maîtrise} \\
%     \hline
%     Compréhension des notions de base des capteurs et de l'acquisition de données & Comprendre les principes fondamentaux des capteurs, de la physique, de l'électronique et de la métrologie pour l'acquisition de données. & Avancé \\
%     \hline
%     Fabrication de capteurs à base de nanoparticules & Être capable de réaliser un capteur de gaz en utilisant des outils de microélectronique : synthèse chimique, intégration et tests. & Intermédiaire \\
%     \hline
%     Utilisation des outils de microélectronique en salle blanche & Respecter les protocoles et manipuler des équipements pour des étapes telles que la photolithographie, la métallisation et l'intégration des nanoparticules. & Avancé \\
%     \hline
%     Etre capable de réaliser une datasheet & Savoir rédiger une datasheet pour un capteur de gaz réalisé, incluant les caractéristiques électriques, et les conditions de test. & Intermédiaire \\
%     \hline
%     \end{tabular}
%     \caption{Compétences acquises lors de la semaine au laboratoire AIME}
% \end{table}

% Ne venant pas d'une formation de physicien ou de chimiste, je ne peux pas dire que je maîtrise complètement le sujet. Mais je peux dire qu'après cette semaine au laboratoire, je serais capable de suivre un protocole de fabrication, et de caractériser un capteur.

% \subsection{Analysis and remarks}
% \indent \indent Etant en alternance, cette semaine à l'AIME a été pour moi une vrai découverte du monde de la recherche et plus particulièrement de la microelectronique. J'ai pu découvrir et manipuler sur des étapes de fabrication de dispositifs microélectroniques, allant de la photolithographie à l'intégration de nanoparticules.

% L'organisation en groupe de 4 m'a également permis d'appréhender le travail en équipe dans un context de recherche, en partageant les tâches et en collaborant pour atteindre un objectif commun. Je pense que cette expérience m'a permis de mieux comprendre les enjeux et les défis de la recherche, et des challenges que sont de devoir suivre un protocole expérimental rigoureux, sans se tromper dans les manipulations.

\section{LAB at AIME}

\subsection{Context}

\indent \indent INSA Toulouse has an Interdisciplinary Microelectronics and Electronics Workshop (AIME), which is a research and training laboratory in microelectronics and electronics. As part of the ISS program, we spent a week there to build a gas sensor through the chemical synthesis of $WO_3$ nanoparticles and their integration into microelectronic devices. This lab is equipped with clean rooms that allow high-precision procedures, essential for micro-scale fabrication.

However, the initial fabrication steps, up to photolithography step 2 (“Contacts Opening”), had already been carried out by the lab technicians, allowing us to focus on the subsequent stages.

\subsection{Technical Summary}

The project aimed to produce a gas sensor based on $WO_3$ nanoparticles, a semiconductor material known for its gas sensitivity. The main steps were, in order:

\begin{enumerate}
    \item Carry out photolithography step 3 for the aluminum electrode etching.
    \item Assemble the devices, including cutting and mounting the chips on TO5 supports.
    \item Synthesize $WO_3$ nanoparticles, involving seed preparation and nanowire growth via hydrothermal processes.
    \item Integrate $WO_3$ nanorods onto interdigitated combs using dielectrophoresis.
    \item Perform electrical characterization of the sensors in a controlled atmosphere containing ethanol or ammonia.
\end{enumerate}

\subsection{Practical Work}

During the week, we were able to complete steps 1 through 5 of the lab. We worked in groups of four, and each group was responsible for its own sensor. I was in charge of synthesizing the $WO_3$ nanoparticles and integrating them onto the interdigitated combs. I also participated in the electrical characterization of the sensors.

\subsection{Skills Acquired}

\indent \indent Thanks to this lab, I believe I have acquired the following skills:

\begin{table}[H]
    \centering
    \begin{tabular}{|p{5cm}|p{8cm}|p{3cm}|}
    \hline
    \textbf{Skill} & \textbf{Description} & \textbf{Level of Mastery} \\
    \hline
    Understanding Basic Concepts of Sensors and Data Acquisition & 
    Understanding the fundamental principles of sensors, physics, electronics, and metrology for data acquisition. & 
    Advanced \\
    \hline
    Fabrication of Nanoparticle-based Sensors & 
    Being able to build a gas sensor using microelectronics tools: chemical synthesis, integration, and testing. & 
    Intermediate \\
    \hline
    Using Microelectronics Tools in Clean Rooms & 
    Following protocols and handling equipment for steps such as photolithography, metallization, and nanoparticle integration. & 
    Advanced \\
    \hline
    Creating a Datasheet & 
    Knowing how to write a technical datasheet for a gas sensor, including electrical characteristics and test conditions. & 
    Intermediate \\
    \hline
    \end{tabular}
    \caption{Skills acquired during the week at the AIME laboratory}
\end{table}

Not coming from a physicist or chemist background, I can't say I've completely mastered the subject. However, after this week in the lab, I feel capable of following a fabrication protocol and characterizing a sensor.

\subsection{Analysis and Remarks}

\indent \indent As a work-study student, this week at AIME was a real discovery for me of the research environment, and more specifically of microelectronics. I was able to explore and handle various steps in the fabrication of microelectronic devices, ranging from photolithography to nanoparticle integration.

Working in groups of four also allowed me to understand teamwork in a research context, by sharing tasks and collaborating to achieve a common goal. I believe this experience has helped me to better grasp the challenges of research as well as the importance of following a strict experimental protocol without making errors during the procedures.
