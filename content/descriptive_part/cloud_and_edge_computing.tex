\section{Cloud and Edge Computing}
\subsection{Context}
%Le cours sur le Cloud and Edge Computing a été le premier cours que j'ai eu en 5ISS. Il introduit les concepts clés de l'informatique distribuée et omniprésente. Il met en lumière des architectures telles que le « Service Oriented Computing (SOC) », la virtualisation, ou encore le rôle des technologies cloud dans le développement de l'informatique omniprésente et utilitaire.
\paragraph{}Among the first few courses I came across during my final year at 5ISS was Cloud and Edge Computing. This course introduced me to core concepts of both distributed and pervasive computing together with the analysis of a basic architecture that included techniques of Service-Oriented Computing-SOC, virtualization techniques, and cloud technologies as enablers of utility and ubiquitous computing. By nature, the course had both a theoretical and hands-on combination that gave an insight into how such paradigms shape the current IT systems and their applications.

\subsection{Technical Summary} % = synthèse du cours
%- Virtualization: Concepts, hypervisors (Type 1 and 2), paravirtualization, containerization, and VM templates. \\
%Comparison between "Service Orchestration" and "Service Choregraphy". -> Which is the best? - > It depends on the context: \\
%The orchestrator slows down the system but is easier to update.  \\
%- > Depends on the development model and on the economic model. \\
%- Cloud Computing: NIST and other definitions, core characteristics such as on-demand service, scalability, pay-per-use models, and service delivery (IaaS, PaaS, SaaS). \\
%- Edge Computing: Integration with IoT, the distribution of data control, benefits (e.g., latency reduction, data sovereignty), and challenges like connectivity and limited processing power. \\
%- Fog Computing: Bridging cloud services closer to data sources for improved real-time interaction and reduced data travel distance. "Cloudifying" the IOT world !!!

\paragraph{}Virtualization has been amply covered in class. To be more specific, we discovered the different types of it: hypervisors, Type 1 and 2, paravirtualization, and containerization. We have also considered Virtual Machine Templates, which strongly ease virtual environments management and deployment. A specific highlight was the comparison between service orchestration and service choreography, where we analyzed the strengths and limitations of each. While orchestration does indeed ease updates by adopting a centralized approach, it tends to make the system slower. On the other hand, choreography is decentralized, enhancing autonomy but with a very painstaking design process. The choice between them is necessarily dependent on the specific contexts of development and economics in which they are applied.\\

Another important area discussed in this course was cloud computing: a discussion of definitions, among them the widely recognized framework from NIST, but also established its core characteristics. On-demand service provision, scalability, pay-per-use, and supply of services in models like IaaS, PaaS, and SaaS-all these form key elements that make cloud computing. We also looked at edge computing, which is integral to the IoT ecosystem, and allows for latency reduction and ensures data sovereignty. However, this also presents challenges of connectivity and the processing power of the edge devices. Finally, we discussed fog computing: an intermediary layer that connects cloud services with data sources, reducing the distance data needs to travel, and allowing real-time interaction. This was aptly referred to as the process of "cloudifying the IoT world."

\subsection{Practical work}
%La partie pratique s'est organisée en 2 TP, un sur la découverte des Cloud Hypervisors et l'autre sur l'orchestration de services dans un environnement hybride cloud/edge. Ces TP m'ont permis d'approfondir mes connaissances des concepts et des technologies utilisées dans les différentes techniques de virtualisation.
\paragraph{}The practical work for this course was divided into two major sessions, each designed to reinforce the theoretical aspects covered in class. In the first session, we explored cloud hypervisors by engaging with tools such as VirtualBox and the OpenStack API. This practical session allowed me to experiment with various virtualization techniques and understand their implementation in controlled environments. The second session was on service orchestration in hybrid cloud and edge environments. In that exercise, I learned the deployment and management of a distributed system operating across these two architectures, thus providing insights into their complexities and potential.
\paragraph{}These practical sessions were crucial for bridging the gaps in theory into applications. These have provided a chance not only to comprehend the conceptual aspects but also to apply the same in realistic scenarios. This solidifies my comprehension and confidence in handling Hybrid Architectures.

\subsection{Competences gained}
%- Proficiency in cloud and edge computing architecture design. \\
%- Skills in using virtualization tools and managing virtual environments (VMs and containers). \\
%- Expertise in deploying services using container orchestration platforms like Kubernetes. \\
%- The ability to design resilient and low-latency systems tailored for IoT and distributed scenarios. \\
%- Insight into handling edge node variability, ensuring service continuity despite real-time disruptions.
\paragraph{}The course really enhanced my technical capability and comprehension of Distributed Systems. Cloud and Edge computing architectures focusing on high performance and dependability. My experience with VirtualBox and OpenStack gave me hands-on experience in creating and managing virtual environments: virtual machines and containers. This also includes the development of expertise in deploying and orchestrating containerized services through the use of platforms like Kubernetes, which is critical in modern-day designing of distributed systems. The course enhanced my system design skills for low latency and high resilience, particularly in IoT and Distributed Environments. Apart from that, I learned real-time disruption management at the edge and continuity of service, despite variability in edge nodes.

\subsection{Analysis and remarks}
%Le format des TP est pour moi très intérressant car étant divisé en une partie théorique, permettant de fixer les concepts essentiels du cours en début de TP, accompagné d'une partie pratique, manipulation Virtual Box et l'API OpenStack, tout en permettant de mettre en application ces concepts de cours.
\paragraph{}One of the strong points of this course is the structuring of the practical sessions, which did combine theory and practice quite nicely. Each session started with a theoretical introduction to reinforce the central concepts of the course; we then had to go through an implementation phase by putting these ideas into practice with tools such as VirtualBox and OpenStack APIs. In this way, both my understanding was solidly shaped and practical skills were gained; hence, learning has been both engaging and very effective. I found this format particularly enriching because it mirrors real-world workflows, where a deep understanding of concepts must complement practical expertise.

\subsection{Reflections}
\paragraph{}This course gave me a better understanding of the interaction between cloud and edge computing systems and how they enable innovative, scalable, and efficient solutions. The smooth integration of theoretical and practical aspects has prepared me for the challenges of designing and managing distributed systems in professional life. In this learning process, I developed not only my technical skills but also the problem-solving approach needed to handle modern IT system complexities.