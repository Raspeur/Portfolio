\section{Energy for connected objects}
\subsection{Context}
\paragraph{}The "Energies for Connected Objects" course dealt with pioneering strategies to feed IoT devices without batteries or wired systems. This training focused on Ambient Energy Harvesting and Wireless Power Transfer applied to the high demand for energy-autonomous solutions in healthcare, smart cities, and environmental monitoring, among others. It brought together a multidisciplinary approach, from theoretical insights to efficient, sustainable, and reliable energy system design.

\subsection{Technical}
\paragraph{}It contained discussions on capacitors and supercapacators to buffer energy, wireless power transfer that has been highly evolved with a near-field mechanism by means of capacitive and inductive coupling to a far-field radiative by the use of a rectenna, and energy harvesting techniques using ambient energy through conversion of energy due to light, mechanical motion, thermal gradient, and electromagnetic waves. The course also integrated optimization strategies, such as antenna design to match the frequency and software optimizations that would reduce energy consumption in connected devices.

\subsection{Practical work}
\paragraph{}The practical work for this course involved the design and testing of energy systems in the laboratory. One of the projects was to power an LED from ambient RF energy. This involved analyzing a rectifier circuit, choosing appropriate antennas, and optimizing the energy transfer for efficient operation. I used GNURadio and spectrum analyzers to measure RF power output, find the optimal operating frequencies, and test various energy storage configurations. The "store then use" strategy, as realized with a bq25504 power management unit and a TPS63031 DC-DC converter, allowed for efficient energy buffering and utilization in low-power scenarios.

\subsection{Skills acquired}
\paragraph{}In this course, I developed high-level technical expertise in low-power circuit design, energy harvesting systems, and wireless power transfer technologies. I acquired hands-on experience with laboratory tools and methodologies such as frequency sweeps, impedance matching, and antenna characterization. Beyond the technical skills, I learned to interpret complex system data and optimize designs for practical applications. The course also fostered innovative problem-solving skills, particularly in balancing theoretical frameworks with real-world constraints.

\subsection{Analysis and remarks}
\paragraph{}This class demonstrated the potentials and pitfalls of designing energy-autonomous IoT devices. Even though I implemented various systems, such as the energy harvesting LED with ease, environmental inputs had high variability that posed the biggest challenges. For instance, supplying power continuously in constantly fluctuating electromagnetic environments forced me to devise creative solutions but also highlighted the necessity of using hybrid energy systems. Furthermore, a central design challenge was determining the balance between optimizing the capture of energy and reliability of the system.

\subsection{Reflections}
\paragraph{}"Energy for Connected Objects" provided a great platform for innovative energy solution development within the IoT domain. It identified the key challenges arising from future technologies in relevance to sustainability and efficiency; therefore, this was focused on the latest energy system approaches and applications. In the future, my skills and knowledge will guide my work in energy-autonomous devices, mainly in scalable wireless sensor networks, but also including advanced material development for energy conversion and storage. These experiences crystallized my aspiration to further the frontier of IoT energy systems toward sustainable technological development.